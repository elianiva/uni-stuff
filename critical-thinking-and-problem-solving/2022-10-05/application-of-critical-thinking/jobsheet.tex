\documentclass[12pt,titlepage]{article}
\usepackage[margin=1.25in]{geometry}
\usepackage{graphicx,amsmath,blindtext,longtable,tabu,tabularx,enumitem}
\usepackage[table]{xcolor}

%% Variables definition
\newcommand{\vSubject}{Critical Thinking and Problem Solving}
\newcommand{\vSubtitle}{Application of Critical Thinking}
\newcommand{\vName}{Dicha Zelianivan Arkana}
\newcommand{\vNIM}{2241720002}
\newcommand{\vClass}{1i}
\newcommand{\vDepartment}{Information Technology}
\newcommand{\vStudyProgram}{D4 Informatics Engineering}

%% [START] Tikz related stuff
\usepackage{tikz}
\usetikzlibrary{svg.path,calc,shapes.geometric,shapes.misc}
\tikzstyle{terminator} = [rectangle, draw, text centered, rounded corners = 1em, minimum height=2em]
\tikzstyle{preparation} = [chamfered rectangle, chamfered rectangle sep=0.75em, draw, text centered, minimum height = 2em]
\tikzstyle{process} = [rectangle, draw, text centered, minimum height=2em]
\tikzstyle{decision} = [diamond, aspect=2, draw, text centered, minimum height=2em]
\tikzstyle{data}=[trapezium, draw, text centered, trapezium left angle=60, trapezium right angle=120, minimum height=2em]
\tikzstyle{connector} = [line width=0.25mm,->]
%% [END] Tikz related stuff

%% [START] Fancy header related stuff
\usepackage{fancyhdr}
\pagestyle{fancy}
\setlength{\headheight}{15pt} % compensate fancyhdr style
\fancyhead{}
\fancyfoot{}
\fancyfoot[L]{\thepage}
\fancyfoot[R]{\textit{\vSubject - \vSubtitle}}
\renewcommand{\footrulewidth}{0.4pt}% default is 0pt, overline for footer
%% [END] Fancy header related stuff

%% [START] Custom tabular command related stuff
\usepackage{tabularx}
\newcommand{\details}[2]{
    #1 & #2  \\
}
%% [END] Custom tabular command related stuff

%% [START] Figure related stuff
\newcommand{\image}[3][1]{
    \begin{figure}[h]
        \centering
        \includegraphics[#1]{#2}
        \caption{#3}
        \label{#3}
    \end{figure}
}
%% [END] Figure related stuff

\newcommand{\y}{\cellcolor{yellow!75}}

\begin{document}
\begin{titlepage}
    \centering
    \vfill
    {\bfseries\LARGE
        \vSubject\\
        \vskip0.25cm
        \vSubtitle
    }
    \vfill
    \includegraphics[width=6cm]{images/polinema-logo.png}
    \vfill
    {
        \textbf{Name}\\
        \vName\\
        \vskip0.5cm
        \textbf{NIM}\\
        \vNIM\\
        \vskip0.5cm
        \textbf{Class}\\
        \vClass\\
        \vskip0.5cm
        \textbf{Department}\\
        \vDepartment\\
        \vskip0.5cm
        \textbf{Study Program}\\
        \vStudyProgram
    }
\end{titlepage}

\section*{Task 1}
The robot beaver can multitask. Each task requires 1, 2, 3, or more hours of work. In one hour, the robot can only do one
task. At the end of each hour, he checks if there is a new task:
\begin{enumerate}
    \item If yes, then the robot must start working on the new task.
    \item If not, the robot continues to do the task that has not been done for the longest time.
\end{enumerate}

The following is an example of a work schedule for the robot in a day.

\begin{itemize}
    \item At 8:00, there is a task that takes 7 hours
    \item At 10:00, comes the task that takes 3 hours
    \item At 12:00 o'clock, comes the task that takes 5 hours
\end{itemize}

In the table, the yellow color indicates the task is in progress, the white color indicates the task is pending.

\begin{figure}[h]
    \centering
    \includegraphics*[width=\textwidth]{./images/robot-table.png}
\end{figure}

If the robot accepts the following four tasks:

\begin{itemize}
    \item Task-1: at 8:00 p.m. takes 5 hours
    \item Task-2: at 11:00 takes 3 hours
    \item Task-3: at 14:00 takes 5 hours
    \item Task-4: at 17:00 takes 2 hours
\end{itemize}

\pagebreak

At what time will each task be completed. Robot can do only one task at once time.

\begin{table}[h]
    \caption{Robot Schedule}
    \begin{tabularx}{\textwidth}{|l|c|c|c|c|c|c|c|c|c|c|c|c|c|c|c|}
        \hline
        \textbf{Task} & \textbf{8} & \textbf{9} & \textbf{10} & \textbf{11} & \textbf{12} & \textbf{13} & \textbf{14} & \textbf{15} & \textbf{16} & \textbf{17} & \textbf{18} & \textbf{19} & \textbf{20} & \textbf{21} & \textbf{22} \\
        \hline
        Task 1 & \y & \y & \y &   & \y &   &   & \y &   &   &   &   &   &   &   \\
        \hline
        Task 2 &   &   &   & \y &   & \y &   &   & \y &   &   &   &   &   &   \\
        \hline
        Task 3 &   &   &   &   &   &   & \y &   &   &   & \y &   & \y & \y & \y \\
        \hline
        Task 4 &   &   &   &   &   &   &   &   &   & \y &   & \y &   &   &   \\
        \hline
    \end{tabularx}
\end{table}

\section*{Task 2}

During their 6-day vacation, Laravel and Zend have a plan to go to Grandma's village. Incidentally, there were three
farmers A, B, and C who needed help in cultivating their respective fields. They offered Laravel and Zend a fee if they
would help them. Each of these farmers makes a different offer:
\begin{itemize}
    \item Farmer A offers 10 thousand rupiah for each (Laravel and Zend) every day.
    \item {
        Farmer B will only give Zend ten thousand rupiah on the first day then each subsequent increase by 10 thousand to 20
        thousand, 30 thousand, and so on, while he will give Laravel on the first day 100 thousand rupiah and then decrease 10
        thousand rupiah every following day to 90 thousand, 80 thousand, and so on.
    }
    \item {
        Farmer C is not interested in Zend's help, so he will only give 1 thousand rupiah on the first day and will not give
        anything on the next day. As for Laravel, he will give a thousand rupiah on the first day, then every next day double
        from the previous. So, Laravel will get a thousand rupiah, 2 thousand rupiah, 4 thousand rupiah, 8 thousand rupiah and
        so on. They intend to spend every day of their holiday in grandmother's village helping the farmer, and they both have
        promised to work for the same farmer. Regarding wages, they have also secretly agreed to share equally what they get.
    }
\end{itemize}
\pagebreak
For which farmers do they work so that they can get the most fee?

\setcounter{table}{1}
\begin{table}[h]
    \caption{Farmer's Fee}
    \begin{tabularx}{\textwidth}{|l|c|c|c|c|c|c|c|}
        \hline
        \textbf{Farmer} & \textbf{Day 1} & \textbf{Day 2} & \textbf{Day 3} & \textbf{Day 4} & \textbf{Day 5} & \textbf{Day 6} & \textbf{Total} \\
        \hline
        Laravel - Farmer 1 & 10k & 10k & 10k & 10k & 10k & 10k & 60k \\ \hline
        Zend - Farmer 1    & 10k & 10k & 10k & 10k & 10k & 10k & 60k \\ \hline
        Laravel - Farmer 2 & 100k & 90k & 80k & 70k & 60k & 60k & 450k \\ \hline
        Zend - Farmer 2    & 10k & 20k & 30k & 40k & 50k & 60k & 210k \\ \hline
        Laravel - Farmer 3 & 1k & 2k & 4k & 8k & 16k & 32k & 63k \\ \hline
        Zend - Farmer 3    & 1k & 0 & 0 & 0 & 0 & 0 & 0 \\ \hline
    \end{tabularx}
\end{table}

Total:
\begin{itemize}
    \item Farmer 1 = 60k + 60k = 120k
    \item Farmer 2 = 450k + 210k = 660k
    \item Farmer 3 = 63k + 1k = 64k
\end{itemize}

Laravel and Zend can get the most profit if they work for Farmer 2

\section*{Task 3}
The famous blue colored diamond has been stolen from the museum. The thief success to exchange the diamond for cheap
green imitation jewellery. The diamond exhibition today was attended by 200 visitors. The visitors entered the exhibition
room one by one. The Java inspector should be able to catch the thief by interrogating some of the visitors. Inspector Java
has a list of the names of the 200 visitors who entered the exhibition hall today. The Java inspector will ask everyone the
same question: Was the diamond green or blue when you saw it? Every visitor will answer honestly; except the thief, who
will answer the color of the diamond is green. The Java inspector is very smart and will use a strategy where the number
of people to be asked question will be minimal.

\pagebreak

Which of the following statements can the Java Inspector deliver without lying? Explain

\begin{enumerate}[label=\alph*)]
    \item This task is a difficult one; I need to ask at least 200 people, but the most likely is 199 people.
    \item I can't promise anything. If I am unlucky, then I will question every visitor.
    \item It's not enough to just ask 10 people (unless I'm lucky) but I believe I can get my job done by asking less than 200 people.
    \item I can guarantee that I can find the thief by simply asking less than 10 people
\end{enumerate}

\textbf{Explanation:} The answer is \textbf{C}. Assuming the visitors enter the room one by one and we have the time series data, we can ask every visitor starting with the first visitor who enters the room.
Every visitors will answer that the diamond is blue until the diamond is replaced.
We can infer that the first visitor who answers green is the thief because he is the one who replaced it first.
This way, we won't have to ask 200 people.

\end{document}

