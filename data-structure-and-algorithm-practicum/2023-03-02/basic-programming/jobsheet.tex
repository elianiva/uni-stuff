\documentclass[12pt,titlepage]{article}
\usepackage[margin=1.25in]{geometry}
\usepackage{graphicx,amsmath,blindtext,minted}

%% Variables definition
\newcommand{\vSubject}{Data Structure and Algorithm}
\newcommand{\vSubtitle}{Basic Programming}
\newcommand{\vName}{Dicha Zelianivan Arkana}
\newcommand{\vNIM}{2241720002}
\newcommand{\vClass}{1i}
\newcommand{\vDepartment}{Information Technology}
\newcommand{\vStudyProgram}{D4 Informatics Engineering}

%% [START] Tikz related stuff
\usepackage{tikz}
\usetikzlibrary{svg.path,calc,shapes.geometric,shapes.misc}
\tikzstyle{terminator} = [rectangle, draw, text centered, rounded corners = 1em, minimum height=2em]
\tikzstyle{preparation} = [chamfered rectangle, chamfered rectangle sep=0.75em, draw, text centered, minimum height = 2em]
\tikzstyle{process} = [rectangle, draw, text centered, minimum height=2em]
\tikzstyle{decision} = [diamond, aspect=2, draw, text centered, minimum height=2em]
\tikzstyle{data}=[trapezium, draw, text centered, trapezium left angle=60, trapezium right angle=120, minimum height=2em]
\tikzstyle{connector} = [line width=0.25mm,->]
%% [END] Tikz related stuff

%% [START] Fancy header related stuff
\usepackage{fancyhdr}
\pagestyle{fancy}
\setlength{\headheight}{15pt} % compensate fancyhdr style
\fancyhead{}
\fancyfoot{}
\fancyfoot[L]{\thepage}
\fancyfoot[R]{\textit{\vSubject - \vSubtitle}}
\renewcommand{\footrulewidth}{0.4pt}% default is 0pt, overline for footer
%% [END] Fancy header related stuff

%% [START] Custom tabular command related stuff
\usepackage{tabularx}
\newcommand{\details}[2]{
    #1 & #2  \\
}
%% [END] Custom tabular command related stuff

%% [START] Figure related stuff
\newcommand{\image}[3][1]{
    \begin{figure}[h]
        \centering
        \includegraphics[#1]{#2}
        \caption{#3}
        \label{#3}
    \end{figure}
}
%% [END] Figure related stuff

\begin{document}
\begin{titlepage}
    \centering
    \vfill
    {\bfseries\LARGE
        \vSubject\\
        \vskip0.25cm
        \vSubtitle
    }
    \vfill
    \includegraphics[width=6cm]{images/polinema-logo.png}
    \vfill
    {
        \textbf{Name}\\
        \vName\\
        \vskip0.5cm
        \textbf{NIM}\\
        \vNIM\\
        \vskip0.5cm
        \textbf{Class}\\
        \vClass\\
        \vskip0.5cm
        \textbf{Department}\\
        \vDepartment\\
        \vskip0.5cm
        \textbf{Study Program}\\
        \vStudyProgram
    }
\end{titlepage}

\section*{Questions}
\begin{enumerate}
    \item {
        \begin{minted}[autogobble,fontsize=\small]{java}
            import java.util.Scanner;

            public class ConditionalProblem {
                static Scanner input = new Scanner(System.in);
                static double ASSIGNMENT_FACTOR = 0.2;
                static double MIDTERM_FACTOR = 0.35;
                static double FINAL_EXAM_FACTOR = 0.45;

                public static void main(String[] args) {

                    System.out.println("Final Score Calculator");
                    System.out.println("======================");
                    int assignmentScore = getIntValue("Input your assignemnt score: ");
                    int midTermScore = getIntValue("Input your midterm score: ");
                    int finalExamScore = getIntValue("Input your final exam score: ");
                    System.out.println("======================");
                    double finalScore = (ASSIGNMENT_FACTOR * assignmentScore)
                            + (MIDTERM_FACTOR * midTermScore)
                            + (FINAL_EXAM_FACTOR * finalExamScore);
                    String convertedScore = convertScore(finalScore);
                    System.out.printf("Final score: %.2f", finalScore);
                    System.out.printf("Grade: %s", convertedScore);
                    System.out.println("======================");
                    if (convertedScore.equals("A")
                            || convertedScore.equals("B+")
                            || convertedScore.equals("B")
                            || convertedScore.equals("C+")
                            || convertedScore.equals("C+")) {
                        System.out.println("Congratulations! You passed!");
                    } else {
                        System.out.println("Unfortunately you failed.");
                    }
                }

                private static int getIntValue(String prompt) {
                    while (true) {
                        System.out.print(prompt);
                        String value = input.next();
                        if (!value.isEmpty()) {
                            int intValue = Integer.parseInt(value);
                            if (intValue >= 0 && intValue <= 100) {
                                return intValue;
                            }
                        }
                        System.out.println(
                            "Please insert the correct value, must be from 0 to 100!"
                        );
                    }
                }

                private static String convertScore(double score) {
                    if (score > 80 && score <= 100) return "A";
                    if (score > 73 && score <= 80) return "B+";
                    if (score > 65 && score <= 73) return "B";
                    if (score > 60 && score <= 65) return "C+";
                    if (score > 50 && score <= 60) return "C";
                    if (score > 39 && score <= 50) return "D";
                    return "E";
                }
            }
        \end{minted}
    }
    \item {
        \begin{minted}[autogobble,fontsize=\small]{java}
            import java.util.Scanner;

            public class LoopProblem {
                static Scanner input = new Scanner(System.in);
                static int NIM_LIMIT = 10;
                static String[] DAYS = {
                        "SUNDAY",
                        "MONDAY",
                        "TUEDAYS",
                        "WEDNESDAY",
                        "THURSDAY",
                        "FRIDAY",
                        "SATURDAY"
                };

                public static void main(String[] args) {
                    String nim = getNIM("Insert your NIM: ");
                    int repeatAmount = getLastTwoDigit(nim);
                    for (int i = 0; i < repeatAmount; i++) {
                        System.out.printf("%s ", DAYS[i % 7]);
                    }
                }

                private static String getNIM(String prompt) {
                    while (true) {
                        System.out.print(prompt);
                        String nim = input.next();

                        if (nim.length() < NIM_LIMIT) {
                            System.out.println(
                                "Your NIM should be at least have a length of 10"
                            );
                            continue;
                        }

                        // every character should be a number
                        boolean isValid = true;
                        for (int i = 0; i < nim.length(); i++) {
                            if (Character.isAlphabetic(nim.charAt(i))) {
                                isValid = false;
                            }
                        }
                        if (!isValid) {
                            System.out.println("Your NIM cannot contain any alphabet!");
                            continue;
                        }

                        return nim;
                    }
                }

                private static int getLastTwoDigit(String nim) {
                    String lastTwoDigit = nim.substring(nim.length() - 2, nim.length());
                    int value = Integer.parseInt(lastTwoDigit);
                    return value < 10 ? value + 10 : value;
                }
            }
        \end{minted}
    }
    \item {
        \begin{minted}[autogobble,fontsize=\small]{java}
            public class ArrayProblem {
                static String[] FLOWER_KINDS = {
                        "Aglaonema",
                        "Taro",
                        "Alocasia",
                        "Rose"
                };
                static int[][] STOCK_BY_BRANCH = {
                        // Aglaonema - Taro - Alocasia - Rose
                        { 10, 5, 15, 7 }, // Royal Garden 1
                        { 6, 11, 9, 12 }, // Royal Garden 2
                        { 2, 10, 10, 5 }, // Royal Garden 3
                        { 5, 7, 12, 9 }, // Royal Garden 4
                };
                static int[] FLOWER_PRICES = {
                        75_000, // Aglaonema
                        50_000, // Taro
                        60_000, // Alocasia
                        10_000 // Rose
                };

                public static void main(String[] args) {
                    int[] branchesStock = countStockAcrossBranches(STOCK_BY_BRANCH);
                    for (int flowerId = 0; flowerId < branchesStock.length; flowerId++) {
                        System.out.printf(
                                "Stock for %s: %d\n",
                                FLOWER_KINDS[flowerId],
                                branchesStock[flowerId]);
                    }
                    System.out.println("====================");
                    int income = countIncomeForBranch(0, new int[] { 1, 2, 5, 0 });
                    System.out.printf("Income for Royal Garden 1 is: %d", income);
                }

                private static int[] countStockAcrossBranches(int[][] stock) {
                    int[] branchesStock = new int[4];
                    for (int branchId = 0; branchId < stock.length; branchId++) {
                        int branchStocks = stock[branchId].length;
                        for (int flowerId = 0; flowerId < branchStocks; flowerId++) {
                            branchesStock[branchId] += stock[branchId][flowerId];
                        }
                    }
                    return branchesStock;
                }

                private static int countIncomeForBranch(int branchId, int[] lossDetail) {
                    if (lossDetail.length != FLOWER_KINDS.length) {
                        System.out.println(
                            "Loss detail can't be less than the types of the flower"
                        );
                        System.exit(1);
                    }
                    int income = 0;
                    for (int stock : STOCK_BY_BRANCH[branchId]) {
                        for (int flowerId = 0; flowerId < lossDetail.length; flowerId++) {
                            int flowerIncome = 
                                (stock - lossDetail[flowerId]) * FLOWER_PRICES[flowerId];
                            income += flowerIncome;
                        }
                    }
                    return income;
                }
            }
        \end{minted}
    }
    \item {
        \begin{minted}[autogobble,fontsize=\small]{java}
            public class Fibonacci {
                public static void main(String[] args) {
                    String recursiveFibonacciRow = getFibonacciRow("recursive", 9);
                    System.out.println("Fibonacci using recursion: " + recursiveFibonacciRow);

                    String loopFibonacciRow = getFibonacciRow("loop", 9);
                    System.out.println("Fibonacci using recursion: " + loopFibonacciRow);
                }

                private static String getFibonacciRow(String type, int limit) {
                    int[] fibonacciNumbers = new int[limit];
                    if (type == "recursive") {
                        for (int i = 0; i < limit; i++) {
                            fibonacciNumbers[i] = recursiveFibonacci(i);
                        }
                    } else if (type == "loop") {
                        for (int i = 0; i < limit; i++) {
                            fibonacciNumbers[i] = loopFibonacci(i);
                        }
                    }

                    String row = "";
                    for (int i = 0; i < fibonacciNumbers.length; i++) {
                        row += fibonacciNumbers[i];
                        if (i != fibonacciNumbers.length - 1) {
                            row += ", ";
                        }
                    }
                    return row;
                }

                private static int recursiveFibonacci(int n) {
                    return n > 1
                            ? recursiveFibonacci(n - 1) + recursiveFibonacci(n - 2)
                            : n;
                }

                private static int loopFibonacci(int n) {
                    int x = 1;
                    int y = 0;
                    int result = 0;

                    while (n > 0) {
                        result = x;
                        x = x + y;
                        y = result;
                        n--;
                    }

                    return result;
                }
            }
        \end{minted}
    }
\end{enumerate}

\section*{Assignment}
\begin{enumerate}
    \item {
        \begin{minted}[autogobble,fontsize=\small]{java}
            public class Laundry {
                static int PRICE = 4_500;
                static double DISCOUNT = 0.05;
                // Ani - Budi - Bina - Cita
                static int[] CUSTOMERS = { 4, 15, 6, 11 };

                public static void main(String[] args) {
                    int income = 0;
                    for (int weight : CUSTOMERS) {
                        income += weight > 10 ? (weight * PRICE) * DISCOUNT : weight * PRICE;
                    }
                    System.out.println("The total income for Smile Laundry is: " + income);
                }
            }
        \end{minted}
    }
    \item {
        \begin{minted}[autogobble,fontsize=\small]{java}
            public class Interest {
                static int INITIAL_BALANCE = 1_000_000;
                static int TARGET_BALANCE = 1_500_000;
                static double INTEREST = 1.02;

                public static void main(String[] args) {
                    int monthCount = 0;
                    int balance = INITIAL_BALANCE;
                    while (balance <= TARGET_BALANCE) {
                        balance *= INTEREST;
                        if (balance >= TARGET_BALANCE) {
                            continue;
                        }
                        monthCount++;
                    };
                    System.out.printf(
                        "The balance reached the target after %d months\n",
                        monthCount
                    );
                }
            }
        \end{minted}
    }
    \item {
        \begin{minted}[autogobble,fontsize=\small]{java}
            import java.util.Scanner;

            public class Menu {
                static Scanner input = new Scanner(System.in);

                public static void main(String[] args) {
                    int chosenMenu = chooseMenu();
                    switch (chosenMenu) {
                        case 1:
                            calculateTriangleArea();
                            break;
                        case 2:
                            calculateRectangleArea();
                            break;
                        case 3:
                            calculateCircleArea();
                            break;
                        default:
                            System.out.println("Invalid Menu");
                            break;
                    }
                }

                public static int chooseMenu() {
                    System.out.println("1. Calculate area of triangle");
                    System.out.println("2. Calculate area of rectangle");
                    System.out.println("3. Calculate area of circle");
                    System.out.print("Choose menu: ");
                    return input.nextInt();
                }

                public static void calculateTriangleArea() {
                    System.out.print("Insert the base width: ");
                    int base = input.nextInt();
                    System.out.print("Insert the height: ");
                    int height = input.nextInt();
                    double area = (base / 2) * height;
                    System.out.printf("The area of the triangle is: %.2f\n", area);
                }

                public static void calculateRectangleArea() {
                    System.out.print("Insert the length: ");
                    int length = input.nextInt();
                    System.out.print("Insert the height: ");
                    int height = input.nextInt();
                    double area = length * height;
                    System.out.printf("The area of the rectangle is: %.2f\n", area);
                }

                public static void calculateCircleArea() {
                    System.out.print("Insert the radius: ");
                    int radius = input.nextInt();
                    double area = Math.PI * radius * radius;
                    System.out.printf("The area of the circle: %.2f\n", area);
                }
            }

        \end{minted}
    }
\end{enumerate}

\end{document}

