\documentclass[12pt,titlepage]{article}
\usepackage[margin=1.25in]{geometry}
\usepackage{graphicx,amsmath,blindtext}

%% Variables definition
\newcommand{\vSubject}{Advanced Web Programming}
\newcommand{\vSubtitle}{Migration}
\newcommand{\vName}{Dicha Zelianivan Arkana}
\newcommand{\vNIM}{2241720002}
\newcommand{\vClass}{2i}
\newcommand{\vDepartment}{Information Technology}
\newcommand{\vStudyProgram}{D4 Informatics Engineering}

%% [START] Tikz related stuff
\usepackage{tikz}
\usetikzlibrary{svg.path,calc,shapes.geometric,shapes.misc}
\tikzstyle{terminator} = [rectangle, draw, text centered, rounded corners = 1em, minimum height=2em]
\tikzstyle{preparation} = [chamfered rectangle, chamfered rectangle sep=0.75em, draw, text centered, minimum height = 2em]
\tikzstyle{process} = [rectangle, draw, text centered, minimum height=2em]
\tikzstyle{decision} = [diamond, aspect=2, draw, text centered, minimum height=2em]
\tikzstyle{data}=[trapezium, draw, text centered, trapezium left angle=60, trapezium right angle=120, minimum height=2em]
\tikzstyle{connector} = [line width=0.25mm,->]
%% [END] Tikz related stuff

%% [START] Fancy header related stuff
\usepackage{fancyhdr}
\pagestyle{fancy}
\setlength{\headheight}{15pt} % compensate fancyhdr style
\fancyhead{}
\fancyfoot{}
\fancyfoot[L]{\thepage}
\fancyfoot[R]{\textit{\vSubject - \vSubtitle}}
\renewcommand{\footrulewidth}{0.4pt}% default is 0pt, overline for footer
%% [END] Fancy header related stuff

%% [START] Custom tabular command related stuff
\usepackage{tabularx}
\newcommand{\details}[2]{
    #1 & #2  \\
}
%% [END] Custom tabular command related stuff

%% [START] Figure related stuff
\newcommand{\image}[3][1]{
    \begin{figure}[h]
        \centering
        \includegraphics[#1]{#2}
        \caption{#3}
        \label{#3}
    \end{figure}
}
%% [END] Figure related stuff

\begin{document}
\begin{titlepage}
    \centering
    \vfill
    {\bfseries\LARGE
        \vSubject\\
        \vskip0.25cm
        \vSubtitle
    }
    \vfill
    \includegraphics[width=6cm]{images/polinema-logo.png}
    \vfill
    {
        \textbf{Name}\\
        \vName\\
        \vskip0.5cm
        \textbf{NIM}\\
        \vNIM\\
        \vskip0.5cm
        \textbf{Class}\\
        \vClass\\
        \vskip0.5cm
        \textbf{Department}\\
        \vDepartment\\
        \vskip0.5cm
        \textbf{Study Program}\\
        \vStudyProgram
    }
\end{titlepage}

\section{Questions}
\begin{enumerate}
    \item {
        In \textbf{Practicum 1 - Step 5}, what is the function of the \texttt{APP\_KEY} in the Laravel \texttt{.env} setting file?

        It is used to encrypt the session and other data in Laravel.
    }
    \item {
        In \textbf{Practicum 1}, how do we generate value for \texttt{APP\_KEY}?

        We can generate the value for \texttt{APP\_KEY} by using the command \texttt{php artisan key:generate}.
    }
    \item {
        In \textbf{Practicum 2.1 - Step 1}, by default how many migration files does Laravel have? and what are the migration files for?

        By default, Laravel has 4 migration files. The migration files are used to create users, password resets, failed jobs, and personal access tokens.
    }
    \item {
        By default, the migration file contains the code \texttt{\$table->timestamps();}, what is the purpose/output of the function?

        The function is used to create two columns, \texttt{created\_at} and \texttt{updated\_at}, which are used to store the date and time when the data was created and updated.
    }
    \item {
        In the Migration File, there is a function \texttt{\$table->id();} What type of data does the function return?

        The function returns an \texttt{unsigned big integer} data type.
    }
    \item {
        What is the difference between the migration results in the \texttt{m\_level} table, between using \texttt{\$table->id();} by using \texttt{\$table->id('level\_id');}?

        The difference is that the first one will create a column named \texttt{id} as the primary key, while the second one will create a column named \texttt{level\_id} as the primary key.
    }
    \item {
        In migrations, what is the \texttt{->unique()} function used for?

        The \texttt{->unique()} function is used to create a unique index on a column.
    }
    \item {
        In \textbf{Practicum 2.2 - Step 2}, why does the \texttt{level\_id} column in the \texttt{m\_user} table use \texttt{\$table->unsignedBigInteger('level\_id')}, while the \texttt{level\_id} column in the \texttt{m\_level} table uses \texttt{\$table->id('level\_id')} ?

        The \texttt{\$table->unsignedBigInteger('level\_id')} is used to create a column named \texttt{level\_id} as a foreign key, while the \texttt{\$table->id('level\_id')} is used to create a column named \texttt{level\_id} as the primary key.
    }
    \pagebreak
    \item {
        In \textbf{Practicum 3 - Step 6}, what is the purpose of the \texttt{Hash} Class? and what does the Hash program code mean \texttt{::make('1234');} ?

        The \texttt{Hash} class is used to hash the password. The \texttt{::make('1234')} code is used to hash the string \texttt{1234}.
    }
    \item {
        In \textbf{Practicum 4 - Step 3/5/7}, in the query builder there is a question mark (?), what is the use of the question mark (?) of these?

        The question mark (?) is used as a placeholder for the value that will be passed to the query.
    }
    \item {
        In \textbf{Practicum 6 - Step 3}, what is the purpose of writing protected code \texttt{\$table = 'm\_user';} and protected \texttt{\$primaryKey = 'user\_id';} ?

        The \texttt{\$table} is used to define the table name, while the \texttt{\$primaryKey} is used to define the primary key of the table.
    }
    \item {
        In your opinion, where is it easier to use in performing CRUD operations to the database (DB Façade / Query Builder / Eloquent ORM) ?

        In my opinion, it is easier to use Eloquent ORM to perform CRUD operations to the database.
        But, it depends on the situation and the needs of the application.
        Most of the time I would much prefer the query builder because it's more maintainable in the long run and easier to debug.
        It doesn't involve a lot of magic and it's easier to understand what's going on.
    }
\end{enumerate}

The code can be found here: \texttt{https://github.com/elianiva/pwl\_pos}

\end{document}

