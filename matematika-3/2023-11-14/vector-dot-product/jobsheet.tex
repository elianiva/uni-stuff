\documentclass[12pt,titlepage]{article}
\usepackage[margin=1.25in]{geometry}
\usepackage{graphicx,amsmath,blindtext}

%% Variables definition
\newcommand{\vSubject}{Matematika 3}
\newcommand{\vSubtitle}{Vector Dot Product}
\newcommand{\vName}{Dicha Zelianivan Arkana}
\newcommand{\vNIM}{2241720002}
\newcommand{\vClass}{2i}
\newcommand{\vDepartment}{Information Technology}
\newcommand{\vStudyProgram}{D4 Informatics Engineering}

%% [START] Tikz related stuff
\usepackage{tikz}
\usetikzlibrary{svg.path,calc,shapes.geometric,shapes.misc}
\tikzstyle{terminator} = [rectangle, draw, text centered, rounded corners = 1em, minimum height=2em]
\tikzstyle{preparation} = [chamfered rectangle, chamfered rectangle sep=0.75em, draw, text centered, minimum height = 2em]
\tikzstyle{process} = [rectangle, draw, text centered, minimum height=2em]
\tikzstyle{decision} = [diamond, aspect=2, draw, text centered, minimum height=2em]
\tikzstyle{data}=[trapezium, draw, text centered, trapezium left angle=60, trapezium right angle=120, minimum height=2em]
\tikzstyle{connector} = [line width=0.25mm,->]
%% [END] Tikz related stuff

%% [START] Fancy header related stuff
\usepackage{fancyhdr}
\pagestyle{fancy}
\setlength{\headheight}{15pt} % compensate fancyhdr style
\fancyhead{}
\fancyfoot{}
\fancyfoot[L]{\thepage}
\fancyfoot[R]{\textit{\vSubject - \vSubtitle}}
\renewcommand{\footrulewidth}{0.4pt}% default is 0pt, overline for footer
%% [END] Fancy header related stuff

%% [START] Custom tabular command related stuff
\usepackage{tabularx}
\newcommand{\details}[2]{
    #1 & #2  \\
}
%% [END] Custom tabular command related stuff

%% [START] Figure related stuff
\newcommand{\image}[3][1]{
    \begin{figure}[h]
        \centering
        \includegraphics[#1]{#2}
        \caption{#3}
        \label{#3}
    \end{figure}
}
%% [END] Figure related stuff

\begin{document}
\begin{titlepage}
    \centering
    \vfill
    {\bfseries\LARGE
        \vSubject\\
        \vskip0.25cm
        \vSubtitle
    }
    \vfill
    \includegraphics[width=6cm]{images/polinema-logo.png}
    \vfill
    {
        \textbf{Name}\\
        \vName\\
        \vskip0.5cm
        \textbf{NIM}\\
        \vNIM\\
        \vskip0.5cm
        \textbf{Class}\\
        \vClass\\
        \vskip0.5cm
        \textbf{Department}\\
        \vDepartment\\
        \vskip0.5cm
        \textbf{Study Program}\\
        \vStudyProgram
    }
\end{titlepage}

\section{Exercise}
Jika $a = 5i + 4j + 2k, b = 4i - 5j + 3k, \text{dan}~c = 2i - j - 2k$
\begin{enumerate}
    \item {
        \begin{enumerate}
            \item {
                Nilai $a.b$ dan cosinus arah dari hasil kali vektor $a \times b$
    
                \begin{align*}
                    a.b &= \left[\begin{matrix}
                        i  &  j &  k \\
                        5i & 4j & 2k \\
                        4i & -5j & 3k \\
                    \end{matrix}\right]\\
                    a.b &= (5i \times 4i) + (4j \times -5j) + (2k \times 3k) \\
                    a.b &= 20i^2 - 20j^2 + 6k^2 \\
                    a.b &= 20(1) - 20(1) + 6(1) \\
                    a.b &= 6 \\
                \end{align*}
    
                \begin{align*}
                    |a| &= \sqrt{5^2 + 4^2 + 2^2} \\
                    |a| &= \sqrt{25 + 16 + 4} \\
                    |a| &= \sqrt{45} \\
                    |a| &= 3\sqrt{5} \\
                \end{align*}
    
                \begin{align*}
                    |b| &= \sqrt{4^2 + (-5)^2 + 3^2} \\
                    |b| &= \sqrt{16 + 25 + 9} \\
                    |b| &= \sqrt{50} \\
                    |b| &= 5\sqrt{2} \\
                \end{align*}
    
                \begin{align*}
                    cos\theta &= \frac{5}{3\sqrt{5}} . \frac{4}{5\sqrt{2}} + \frac{4}{3\sqrt{5}} . \frac{-5}{5\sqrt{2}} + \frac{2}{3\sqrt{5}} . \frac{3}{5\sqrt{2}} \\
                    &= \frac{20}{15\sqrt{10}} + \frac{-20}{15\sqrt{10}} + \frac{6}{15\sqrt{10}} \\
                    &= \frac{20 - 20 + 6}{15\sqrt{10}} \\
                    &= \frac{6}{15\sqrt{10}} \\
                    &= \frac{2}{5\sqrt{10}} \\
                    &= \frac{2}{5\sqrt{10}} \times \frac{\sqrt{10}}{\sqrt{10}} \\
                    &= \frac{2\sqrt{10}}{50} \\
                    &= \frac{\sqrt{10}}{25} \\
                    \\
                    \theta &= cos^{-1}(\frac{\sqrt{10}}{25}) \\
                    &= cos^{-1}(0.1265) \\
                    &= 82.76^{\circ} \\
                \end{align*}
            }
            \pagebreak
            \item {
                Ukuran dan cosinus dari hasil kali vektor $a \times b$ dan juga sudut dimana hasil kali vektor membentuk sudut dengan vektor c
    
                \begin{align*}
                    a \times b &= \left[\begin{matrix}
                        i  &  j &  k \\
                        5i & 4j & 2k \\
                        4i & -5j & 3k \\
                    \end{matrix}\right]\\
                    a \times b &= (4j \times 3k - 2k \times -5j)i - (5i \times 3k - 2k \times 4i)j + (5i \times -5j - 4j \times 4i)k \\
                    a \times b &= (12 + 10)i - (15 - 8)j + (-25 - 16)k\\
                    a \times b &= 22i - 7j - 41k\\
                    cos\theta &= \frac{a \times b . c}{|a \times b| . |c|}\\
                    &= \frac{(22i - 7j - 41k).(2i - j - 2k)}{\sqrt{22^2 + 7^2 + (-41)^2}.\sqrt{2^2 + (-1)^2 + (-2)^2}}\\
                    &= \frac{44 + 7 + 82}{\sqrt{22^2 + 7^2 + (-41)^2}.\sqrt{2^2 + (-1)^2 + (-2)^2}}\\
                    &= \frac{133}{\sqrt{22^2 + 7^2 + (-41)^2}.\sqrt{2^2 + (-1)^2 + (-2)^2}}\\
                    &= \frac{133}{\sqrt{484 + 49 + 1681}.\sqrt{4 + 1 + 4}}\\
                    &= \frac{133}{\sqrt{2214}.\sqrt{9}}\\
                    &= \frac{133}{3\sqrt{246}.3}\\
                    &= \frac{133}{9\sqrt{246}}\\
                    &= 0.9421\\
                    \\
                    \theta &= cos^{-1}(0.9421)\\
                    &= 19.57^{\circ}\\
                \end{align*}
            }
        \end{enumerate}
    }
    \item {
        Website Phising Detection Application Using Support Vector Machine (Svm)\\
        https://dx.doi.org/10.30818/jitu.5.1.4836
    }
\end{enumerate}

\pagebreak

\section{Summary}
Corrector: Yanuar Thaif Chalil Candra\\
Value: 100

\end{document}

