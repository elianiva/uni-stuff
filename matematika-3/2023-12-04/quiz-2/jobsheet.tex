\documentclass[12pt,titlepage]{article}
\usepackage[margin=1.25in]{geometry}
\usepackage{graphicx,amsmath,blindtext}

%% Variables definition
\newcommand{\vSubject}{Matematika 3}
\newcommand{\vSubtitle}{Quiz 2}
\newcommand{\vName}{Dicha Zelianivan Arkana}
\newcommand{\vNIM}{2241720002}
\newcommand{\vClass}{2i}
\newcommand{\vDepartment}{Information Technology}
\newcommand{\vStudyProgram}{D4 Informatics Engineering}

%% [START] Tikz related stuff
\usepackage{tikz}
\usetikzlibrary{svg.path,calc,shapes.geometric,shapes.misc}
\tikzstyle{terminator} = [rectangle, draw, text centered, rounded corners = 1em, minimum height=2em]
\tikzstyle{preparation} = [chamfered rectangle, chamfered rectangle sep=0.75em, draw, text centered, minimum height = 2em]
\tikzstyle{process} = [rectangle, draw, text centered, minimum height=2em]
\tikzstyle{decision} = [diamond, aspect=2, draw, text centered, minimum height=2em]
\tikzstyle{data}=[trapezium, draw, text centered, trapezium left angle=60, trapezium right angle=120, minimum height=2em]
\tikzstyle{connector} = [line width=0.25mm,->]
%% [END] Tikz related stuff

%% [START] Fancy header related stuff
\usepackage{fancyhdr}
\pagestyle{fancy}
\setlength{\headheight}{15pt} % compensate fancyhdr style
\fancyhead{}
\fancyfoot{}
\fancyfoot[L]{\thepage}
\fancyfoot[R]{\textit{\vSubject - \vSubtitle}}
\renewcommand{\footrulewidth}{0.4pt}% default is 0pt, overline for footer
%% [END] Fancy header related stuff

%% [START] Custom tabular command related stuff
\usepackage{tabularx}
\newcommand{\details}[2]{
    #1 & #2  \\
}
%% [END] Custom tabular command related stuff

%% [START] Figure related stuff
\newcommand{\image}[3][1]{
    \begin{figure}[h]
        \centering
        \includegraphics[#1]{#2}
        \caption{#3}
        \label{#3}
    \end{figure}
}
%% [END] Figure related stuff

\begin{document}
\begin{titlepage}
    \centering
    \vfill
    {\bfseries\LARGE
        \vSubject\\
        \vskip0.25cm
        \vSubtitle
    }
    \vfill
    \includegraphics[width=6cm]{images/polinema-logo.png}
    \vfill
    {
        \textbf{Name}\\
        \vName\\
        \vskip0.5cm
        \textbf{NIM}\\
        \vNIM\\
        \vskip0.5cm
        \textbf{Class}\\
        \vClass\\
        \vskip0.5cm
        \textbf{Department}\\
        \vDepartment\\
        \vskip0.5cm
        \textbf{Study Program}\\
        \vStudyProgram
    }
\end{titlepage}

\section*{Quiz 2}

Corrector: \textbf{Yanuar Thaif Chalil Candra}\\
Score: \textbf{97.5}

\subsection*{Material 10}
\begin{enumerate}
    \item {
        What is the difference between vectors and vector spaces?

        Vectors is an entity that has magnitude or length and direction.
        Vector space is a set of vectors that can be added and multiplied by a scalar.
    }
    \item {
        Determine the size of the vector $\vec{PQ} = 5i + 8j + 3k$ and $\vec{RS} = 9i + 4j + 5k$

        \begin{align*}
            \vec{PQ} &= 5i + 8j +3k \\
            |\vec{PQ}| &= \sqrt{5^2 + 8^2 + 3^2} \\
            &= \sqrt{25 + 64 + 9} \\
            &= \sqrt{98} \\
            &= 7\sqrt2
        \end{align*}

        \begin{align*}
            \vec{RS} &= 9i + 4j + 5k \\
            |\vec{RS}| &= \sqrt{9^2 + 4^2 + 5^2} \\ 
            &= \sqrt{81 + 16 + 25} \\
            &= \sqrt{102}
        \end{align*}
    }
    \item {
        Known $\vec{u} = (1,2,3,5)$ and $\vec{v} = (4,7,6,2)$ determine the length of each vector

        \begin{align*}
            \vec{u} &= (1, 2, 3, 5) \\
            ||\vec{u}|| &= \sqrt{1^2 + 2^2 + 3^2 + 5^2} \\ 
            &= \sqrt{39}
        \end{align*}

        \begin{align*}
            \vec{v} &= (4,7,6,2) \\ 
            ||\vec{v}|| &= \sqrt{4^2 + 7^2 + 6^2 + 2^2} \\
            &= \sqrt{105}
        \end{align*}
    }
    \item {
        Known $\vec{u} = (2,1,3,1)$ and $\vec{v} = (3,3,1,2)$ determine the distance between the two vectors.

        \begin{align*}
            \vec{u} &= (2,1,3,1) \\
            \vec{v} &= (3,3,1,2) \\
            ||\vec{u} - \vec{v}|| &= \sqrt{(2-3)^2 + (1-3)^2 + (3-1)^2 + (1-2)^2} \\
            &= \sqrt{1 + 4 + 4 + 1} \\
            &= \sqrt{10}
        \end{align*}
    }
\end{enumerate}

\subsection*{Material 12}
\begin{enumerate}
    \setcounter{enumi}{4}
    \item {
        Determine the direction cosine $[l,m,n]$ from vector $\vec{r} = 2i - 2j + 7k$

        \begin{align*}
            r &= \sqrt{2^2 - 2^2 + 7^2} \\ 
            &= \sqrt{4 + 4 + 49} \\ 
            &= \sqrt{57}
        \end{align*}

        \begin{align*}
            l &= \frac{2}{\sqrt{57}} \\
            m &= \frac{-2}{\sqrt{57}} \\
            n &= \frac{7}{\sqrt{57}}
        \end{align*}
    }
    \item {
        What is the scalar product of two vectors if a=3, b=5 and $ \theta = 45$?

        \begin{align*}
            \vec{a} &= 3 \\
            \vec{b} &= 5 \\
            \theta &= 45 \\
            \vec{a} \cdot \vec{b} &= ab\cos\theta \\
            &= 3 \times 5 \times \cos 45 \\
            &= 15 \times \frac{\sqrt{2}}{2} \\
            &= \frac{15\sqrt{2}}{2}
        \end{align*}
    }
    \item {
        What is the vector product of $p \times q = \left[ \begin{matrix}
            i & j & k \\
            3 & 4 & 2 \\
            1 & 5 & 2
        \end{matrix} \right] $

        \begin{align*}
            p \times q &= \left[ \begin{matrix}
                i & j & k \\
                3 & 4 & 2 \\
                1 & 5 & 2
            \end{matrix} \right] \\
            &= i \left| \begin{matrix}
                4 & 2 \\
                5 & 2
            \end{matrix} \right| - j \left| \begin{matrix}
                3 & 2 \\
                1 & 2
            \end{matrix} \right| + k \left| \begin{matrix}
                3 & 4 \\
                1 & 5
            \end{matrix} \right| \\
            &= i(8 - 10) - j(6 - 2) + k(15 - 4) \\
            &= -2i - 4j + 11k
        \end{align*}
    }
    \item {
        If a = 3i + 2j + 2k and b = 2i +1j + 4k. Determine
        \begin{enumerate}
            \item $a . b$
            \item $a \times b$
        \end{enumerate}

        \begin{enumerate}
            \item {
                \begin{align*}
                    a &= 3i + 2j + 2k \\
                    b &= 2i + 1j + 4k \\
                    a \cdot b &= (3 \times 2) + (2 \times 1) + (2 \times 4) \\
                    &= 6 + 2 + 8 \\
                    &= 16
                \end{align*}
            }
            \item {
                \begin{align*}
                    a \times b &= \left[ \begin{matrix}
                        i & j & k \\
                        3 & 2 & 2 \\
                        2 & 1 & 4
                    \end{matrix} \right] \\
                    &= i \left| \begin{matrix}
                        2 & 2 \\
                        1 & 4
                    \end{matrix} \right| - j \left| \begin{matrix}
                        3 & 2 \\
                        2 & 4
                    \end{matrix} \right| + k \left| \begin{matrix}
                        3 & 2 \\
                        2 & 1
                    \end{matrix} \right| \\
                    &= i(8 - 2) - j(12 - 4) + k(3 - 4) \\
                    &= 6i - 8j - k
                \end{align*}
            }
        \end{enumerate}
    }
\end{enumerate}

\pagebreak

\subsection*{Material 12}

\begin{enumerate}
    \setcounter{enumi}{8}
    \item {
        Determine the angle between vectors:
        \begin{align*}
            p &= 3i + 2j + 2k\\
            q &= 3i - 2j + 2k
        \end{align*}
        \begin{enumerate}
            \item {
                First determine the direction cosine $[l, m, n]$ (for p) and $[l', m', n']$ (for q).
                \begin{align*}
                    p &= 3i + 2j + 2k \\
                    ||p|| &= \sqrt{3^2 + 2^2 + 2^2} \\
                    &= \sqrt{9 + 4 + 4} \\
                    &= \sqrt{17} \\
                    l &= \frac{3}{\sqrt{17}} \\
                    m &= \frac{2}{\sqrt{17}} \\
                    n &= \frac{2}{\sqrt{17}}
                \end{align*}
                \begin{align*}
                    q &= 3i - 2j + 2k \\
                    ||q|| &= \sqrt{3^2 + (-2)^2 + 2^2} \\
                    &= \sqrt{9 + 4 + 4} \\
                    &= \sqrt{17} \\
                    l' &= \frac{3}{\sqrt{17}} \\
                    m' &= \frac{-2}{\sqrt{17}} \\
                    n' &= \frac{2}{\sqrt{17}}
                \end{align*}
            }
            \item {
                Then look for cos $\theta = l.l' + m.m' + n.n'$
                \begin{align*}
                    \cos\theta &= l.l' + m.m' + n.n' \\
                    &= \frac{3}{\sqrt{17}} \times \frac{3}{\sqrt{17}} + \frac{2}{\sqrt{17}} \times \frac{-2}{\sqrt{17}} + \frac{2}{\sqrt{17}} \times \frac{2}{\sqrt{17}} \\
                    &= \frac{9}{17} - \frac{4}{17} + \frac{4}{17} \\
                    &= \frac{9}{17}
                \end{align*}
            }
        \end{enumerate}
    }
    \item {
        What is vector direction ratio?

        Vector direction ratio is the ratio of the direction of a vector to the direction of the x, y, and z axes.
    }
\end{enumerate}

\end{document}

