\documentclass[12pt,titlepage]{article}
\usepackage[margin=1.25in]{geometry}
\usepackage{graphicx,amsmath,minted}

%% Variables definition
\newcommand{\vSubject}{Object Oriented Programming}
\newcommand{\vSubtitle}{Inheritance}
\newcommand{\vName}{Dicha Zelianivan Arkana}
\newcommand{\vNIM}{2241720002}
\newcommand{\vClass}{2i}
\newcommand{\vDepartment}{Information Technology}
\newcommand{\vStudyProgram}{D4 Informatics Engineering}

%% [START] Tikz related stuff
\usepackage{tikz}
\usetikzlibrary{svg.path,calc,shapes.geometric,shapes.misc}
\tikzstyle{terminator} = [rectangle, draw, text centered, rounded corners = 1em, minimum height=2em]
\tikzstyle{preparation} = [chamfered rectangle, chamfered rectangle sep=0.75em, draw, text centered, minimum height = 2em]
\tikzstyle{process} = [rectangle, draw, text centered, minimum height=2em]
\tikzstyle{decision} = [diamond, aspect=2, draw, text centered, minimum height=2em]
\tikzstyle{data}=[trapezium, draw, text centered, trapezium left angle=60, trapezium right angle=120, minimum height=2em]
\tikzstyle{connector} = [line width=0.25mm,->]
%% [END] Tikz related stuff

%% [START] Fancy header related stuff
\usepackage{fancyhdr}
\pagestyle{fancy}
\setlength{\headheight}{15pt} % compensate fancyhdr style
\fancyhead{}
\fancyfoot{}
\fancyfoot[L]{\thepage}
\fancyfoot[R]{\textit{\vSubject - \vSubtitle}}
\renewcommand{\footrulewidth}{0.4pt}% default is 0pt, overline for footer
%% [END] Fancy header related stuff

%% [START] Custom tabular command related stuff
\usepackage{tabularx}
\newcommand{\details}[2]{
    #1 & #2  \\
}
%% [END] Custom tabular command related stuff

%% [START] Figure related stuff
\newcommand{\image}[3][1]{
    \begin{figure}[h]
        \centering
        \includegraphics[#1]{#2}
        \caption{#3}
        \label{#3}
    \end{figure}
}
%% [END] Figure related stuff

\begin{document}
\begin{titlepage}
    \centering
    \vfill
    {\bfseries\LARGE
        \vSubject\\
        \vskip0.25cm
        \vSubtitle
    }
    \vfill
    \includegraphics[width=6cm]{images/polinema-logo.png}
    \vfill
    {
        \textbf{Name}\\
        \vName\\
        \vskip0.5cm
        \textbf{NIM}\\
        \vNIM\\
        \vskip0.5cm
        \textbf{Class}\\
        \vClass\\
        \vskip0.5cm
        \textbf{Department}\\
        \vDepartment\\
        \vskip0.5cm
        \textbf{Study Program}\\
        \vStudyProgram
    }
\end{titlepage}

\section{Questions}
\begin{enumerate}
    \item {
        \textbf{ClassA.java}
        \begin{minted}[autogobble,fontsize=\small]{java}
            public class ClassA {
                public int X;
                public int Y;

                public void getNilai(){
                    System.out.println("nilai x : " + X);
                    System.out.println("nilai y : " + Y);
                }
            }
        \end{minted}
        \textbf{ClassB.java}
        \begin{minted}[autogobble,fontsize=\small]{java}
            public class ClassB extends ClassA {
                public int z;

                public void getNilaiZ(){
                    System.out.println("nilai z : " + z);
                }

                public void getJumlah(){
                    System.out.println("jumlah : " + (X + Y + z));
                }
            }
        \end{minted}
        \textbf{Percobaan1.java}
        \begin{minted}[autogobble,fontsize=\small]{java}
            public class Percobaan1 {
                public static void main(String[] args) {
                    ClassB hitung = new ClassB();
                    hitung.X = 20;
                    hitung.Y = 30;
                    hitung.z = 5;
                    hitung.getNilai();
                    hitung.getNilaiZ();
                    hitung.getJumlah();
                }
            }
        \end{minted}
        \pagebreak
        \textbf{Output}
        \begin{minted}[autogobble,fontsize=\small]{java}
            nilai x : 20
            nilai y : 30
            nilai z : 5
            jumlah : 55
        \end{minted}
    }
    \item {
        It was throwing an error because the \texttt{ClassB} wasn't inheriting the \texttt{ClassA}.
    }
\end{enumerate}

\section{Questions}
\begin{enumerate}
    \item {
        \textbf{ClassA.java}
        \begin{minted}[autogobble,fontsize=\small]{java}
            public class ClassA {
                protected int X;
                protected int Y;

                public void setX(int x) {
                    X = x;
                }

                public void setY(int y) {
                    Y = y;
                }

                public void getNilai(){
                    System.out.println("nilai x : " + X);
                    System.out.println("nilai y : " + Y);
                }
            }
        \end{minted}
        \textbf{ClassB.java}
        \begin{minted}[autogobble,fontsize=\small]{java}
            public class ClassB extends ClassA {
                private int z;

                public void setZ(int z) {
                    this.z = z;
                }

                public void getNilaiZ(){
                    System.out.println("nilai z : " + z);
                }

                public void getJumlah(){
                    System.out.println("jumlah : " + (X + Y + z));
                }
            }
        \end{minted}
        \textbf{Percobaan2.java}
        \begin{minted}[autogobble,fontsize=\small]{java}
            public class Percobaan2 {
                public static void main(String[] args) {
                    ClassB hitung = new ClassB();
                    hitung.setX(20);
                    hitung.setY(30);
                    hitung.setZ(5);
                    hitung.getNilai();
                    hitung.getNilaiZ();
                    hitung.getJumlah();
                }
            }
        \end{minted}
        \textbf{Output}
        \begin{minted}[autogobble,fontsize=\small]{java}
            nilai x : 20
            nilai y : 30
            nilai z : 5
            jumlah : 55
        \end{minted}
    }
    \item {
        Because the access modifier was \texttt{private}, it can't be accessed by the child class.
        We need to change it to \texttt{protected} so that it can be accessed by the child class.
    }
\end{enumerate}

\section{Questions}
\begin{enumerate}
    \item {
        \texttt{ClassA} is the superclass of both \texttt{ClassB} and \texttt{ClassC} while \texttt{ClassB} is the superclass
        of \texttt{ClassC}.
        \texttt{ClassC} is the subclass of both \texttt{ClassB} and \texttt{ClassA} while \texttt{ClassB} is the subclass
        of \texttt{ClassA}.
    }
    \item {
        The constructor will be called from the superclass to the subclass. The \texttt{super()} invocation
        must be done in the first line of the constructor. Which is why it throws an error when
        we put it in the last line of the constructor.
    }
    \item {
        It calls the parent class or the superclass constructor.
    }
\end{enumerate}

\pagebreak

\section{Task}
\begin{itemize}
    \item {
        \textbf{Employee.java}
        \begin{minted}[autogobble,fontsize=\small]{java}
            public class Employee {
                private String nip;
                private String name;
                private String address;

                public Employee(String nip, String name, String address) {
                    this.nip = nip;
                    this.name = name;
                    this.address = address;
                }

                public String getName() {
                    return name;
                }

                public int getSalary() {
                    return 0;
                }
            }
        \end{minted}
    }
    \item {
        \textbf{Lecturer.java}
        \begin{minted}[autogobble,fontsize=\small]{java}
            public class Lecturer extends Employee {
                private int creditsCount;
                private int creditValue;

                public Lecturer(String nip, String name, String address) {
                    super(nip, name, address);
                }

                public void setCreditsCount(int creditsCount) {
                    this.creditsCount = creditsCount;
                }

                public void setCreditValue(int creditValue) {
                    this.creditValue = creditValue;
                }

                @Override
                public int getSalary() {
                    return creditsCount * creditValue;
                }
            }
        \end{minted}
    }
    \item {
        \textbf{Payroll.java}
        \begin{minted}[autogobble,fontsize=\small]{java}
            public class Payroll {
                private final List<Employee> employees = new ArrayList<>();

                public void addEmployee(Employee employee) {
                    employees.add(employee);
                }

                public void printPayroll() {
                    for (Employee employee : employees) {
                        System.out.println(employee.getName() + " " + employee.getSalary());
                    }
                }
            }
        \end{minted}
    }
    \item {
        \textbf{TaskMain.java}
        \begin{minted}[autogobble,fontsize=\small]{java}
            public class TaskMain {
                public static void main(String[] args) {
                    Payroll payroll = new Payroll();
                    Lecturer lecturer = new Lecturer("12345678", "Manusia Bernapas", "Mosque Street");
                    lecturer.setCreditsCount(10);
                    lecturer.setCreditValue(100000);
                    payroll.addEmployee(lecturer);
                    payroll.printPayroll();
                }
            }
        \end{minted}
    }
    \item {
        \textbf{Output}
        \begin{minted}[autogobble,fontsize=\small]{java}
            Manusia Bernapas 1000000
        \end{minted}
    }
\end{itemize}

\end{document}

