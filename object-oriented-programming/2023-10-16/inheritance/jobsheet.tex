\documentclass[12pt,titlepage]{article}
\usepackage[margin=1.25in]{geometry}
\usepackage{graphicx,amsmath,minted}

%% Variables definition
\newcommand{\vSubject}{Object Oriented Programming}
\newcommand{\vSubtitle}{Inheritance}
\newcommand{\vName}{Dicha Zelianivan Arkana}
\newcommand{\vNIM}{2241720002}
\newcommand{\vClass}{2i}
\newcommand{\vDepartment}{Information Technology}
\newcommand{\vStudyProgram}{D4 Informatics Engineering}

%% [START] Tikz related stuff
\usepackage{tikz}
\usetikzlibrary{svg.path,calc,shapes.geometric,shapes.misc}
\tikzstyle{terminator} = [rectangle, draw, text centered, rounded corners = 1em, minimum height=2em]
\tikzstyle{preparation} = [chamfered rectangle, chamfered rectangle sep=0.75em, draw, text centered, minimum height = 2em]
\tikzstyle{process} = [rectangle, draw, text centered, minimum height=2em]
\tikzstyle{decision} = [diamond, aspect=2, draw, text centered, minimum height=2em]
\tikzstyle{data}=[trapezium, draw, text centered, trapezium left angle=60, trapezium right angle=120, minimum height=2em]
\tikzstyle{connector} = [line width=0.25mm,->]
%% [END] Tikz related stuff

%% [START] Fancy header related stuff
\usepackage{fancyhdr}
\pagestyle{fancy}
\setlength{\headheight}{15pt} % compensate fancyhdr style
\fancyhead{}
\fancyfoot{}
\fancyfoot[L]{\thepage}
\fancyfoot[R]{\textit{\vSubject - \vSubtitle}}
\renewcommand{\footrulewidth}{0.4pt}% default is 0pt, overline for footer
%% [END] Fancy header related stuff

%% [START] Custom tabular command related stuff
\usepackage{tabularx}
\newcommand{\details}[2]{
    #1 & #2  \\
}
%% [END] Custom tabular command related stuff

%% [START] Figure related stuff
\newcommand{\image}[3][1]{
    \begin{figure}[h]
        \centering
        \includegraphics[#1]{#2}
        \caption{#3}
        \label{#3}
    \end{figure}
}
%% [END] Figure related stuff

\begin{document}
\begin{titlepage}
    \centering
    \vfill
    {\bfseries\LARGE
        \vSubject\\
        \vskip0.25cm
        \vSubtitle
    }
    \vfill
    \includegraphics[width=6cm]{images/polinema-logo.png}
    \vfill
    {
        \textbf{Name}\\
        \vName\\
        \vskip0.5cm
        \textbf{NIM}\\
        \vNIM\\
        \vskip0.5cm
        \textbf{Class}\\
        \vClass\\
        \vskip0.5cm
        \textbf{Department}\\
        \vDepartment\\
        \vskip0.5cm
        \textbf{Study Program}\\
        \vStudyProgram
    }
\end{titlepage}

\section{Questions}
\begin{enumerate}
    \item {
        The superclass from the case study above is \texttt{Karyawan} and the subclasses are \texttt{Manager} and \texttt{Staff}
    }
    \item {
        We use the keyword \texttt{extend} to inherit a class.
    }
    \item {
        The attributes of the \texttt{Manager} class are:
        \begin{itemize}
            \item name: String (inherited)
            \item alamat: String (inherited)
            \item age: int (inherited)
            \item gender: String (inherited)
            \item salary: int (inherited)
            \item allowance: int
        \end{itemize}
    }
    \item {
        The identifier \texttt{super} is used to refer to the parent class. So, \texttt{super.salary} is used to access the \texttt{salary} attribute from the parent class
        which is the \texttt{Karyawan} class.
    }
    \item {
        It's a hierarchal inheritance because it only inherit from a single class.
    }
\end{enumerate}

\section{Task}
\begin{itemize}
    \item {
        \textbf{Computer.java}
        \begin{minted}[autogobble,fontsize=\footnotesize]{java}
            public class Computer {
                public String brand;
                public int processorSpeed;
                public int ramSize;
                public String processorType;

                public Computer(String brand, int processorSpeed, int ramSize, String processorType) {
                    this.brand = brand;
                    this.processorSpeed = processorSpeed;
                    this.ramSize = ramSize;
                    this.processorType = processorType;
                }

                public void showData() {
                    System.out.println("Brand: " + brand);
                    System.out.println("Processor Speed: " + processorSpeed);
                    System.out.println("RAM Size: " + ramSize);
                    System.out.println("Processor Type: " + processorType);
                }
            }
        \end{minted}
    }
    \item {
        \textbf{Laptop.java}
        \begin{minted}[autogobble,fontsize=\footnotesize]{java}
            public class Laptop extends Computer {
                public String batteryType;

                public Laptop(String brand, int processorSpeed, int ramSize, String processorType, 
                              String batteryType) {
                    super(brand, processorSpeed, ramSize, processorType);
                    this.batteryType = batteryType;
                }

                public void showLaptop() {
                    super.showData();
                    System.out.println("Battery Type: " + batteryType);
                }
            }
        \end{minted}
    }
    \item {
        \textbf{Mac.java}
        \begin{minted}[autogobble,fontsize=\footnotesize]{java}
            public class Mac extends Laptop {
                public String security;

                public Mac(String brand, int processorSpeed, int ramSize, String processorType, 
                           String batteryType, String security) {
                    super(brand, processorSpeed, ramSize, processorType, batteryType);
                    this.security = security;
                }

                public void showMac() {
                    super.showLaptop();
                    System.out.println("Security: " + security);
                }
            }
        \end{minted}
    }
    \item {
        \textbf{Windows.java}
        \begin{minted}[autogobble,fontsize=\footnotesize]{java}
            public class Windows extends Laptop {
                public String feature;

                public Windows(String brand, int processorSpeed, int ramSize, String processorType,
                               String batteryType, String feature) {
                    super(brand, processorSpeed, ramSize, processorType, batteryType);
                    this.feature = feature;
                }

                public void showWindows() {
                    super.showLaptop();
                    System.out.println("Feature: " + feature);
                }
            }
        \end{minted}
    }
    \item {
        \textbf{Pc.java}
        \begin{minted}[autogobble,fontsize=\footnotesize]{java}
            public class Pc extends Computer {
                public int monitorSize;

                public Pc(String brand, int processorSpeed, int ramSize, String processorType, 
                          int monitorSize) {
                    super(brand, processorSpeed, ramSize, processorType);
                    this.monitorSize = monitorSize;
                }

                public void showPc() {
                    super.showData();
                    System.out.println("Monitor Size: " + monitorSize);
                }
            }
        \end{minted}
    }
    \item {
        \textbf{InheritanceTaskMain.java}
        \begin{minted}[autogobble,fontsize=\footnotesize]{java}
            public class InheritanceTaskMain {
                public static void main(String[] args) {
                    Mac mac = new Mac("Apple", 2, 4, "Intel", "Lithium", "Fingerprint");
                    Pc pc = new Pc("Dell", 2, 4, "Intel", 15);
                    Windows windows = new Windows("Microsoft", 2, 4, "Intel", "Lithium", "Touchscreen");

                    mac.showMac();
                    System.out.println("---");
                    windows.showWindows();
                    System.out.println("---");
                    pc.showPc();
                }
            }
        \end{minted}
    }
    \item {
        \textbf{Output}
        \begin{minted}[autogobble,fontsize=\footnotesize]{html}
            Brand: Apple
            Processor Speed: 2
            RAM Size: 4
            Processor Type: Intel
            Battery Type: Lithium
            Security: Fingerprint
            ---
            Brand: Microsoft
            Processor Speed: 2
            RAM Size: 4
            Processor Type: Intel
            Battery Type: Lithium
            Feature: Touchscreen
            ---
            Brand: Dell
            Processor Speed: 2
            RAM Size: 4
            Processor Type: Intel
            Monitor Size: 15
        \end{minted}
    }
\end{itemize}

\end{document}

