\documentclass[12pt,titlepage]{article}
\usepackage[margin=1.25in]{geometry}
\usepackage{graphicx,amsmath,minted}

%% Variables definition
\newcommand{\vSubject}{Object Oriented Programming}
\newcommand{\vSubtitle}{Overloading and Overriding}
\newcommand{\vName}{Dicha Zelianivan Arkana}
\newcommand{\vNIM}{2241720002}
\newcommand{\vClass}{2i}
\newcommand{\vDepartment}{Information Technology}
\newcommand{\vStudyProgram}{D4 Informatics Engineering}

%% [START] Tikz related stuff
\usepackage{tikz}
\usetikzlibrary{svg.path,calc,shapes.geometric,shapes.misc}
\tikzstyle{terminator} = [rectangle, draw, text centered, rounded corners = 1em, minimum height=2em]
\tikzstyle{preparation} = [chamfered rectangle, chamfered rectangle sep=0.75em, draw, text centered, minimum height = 2em]
\tikzstyle{process} = [rectangle, draw, text centered, minimum height=2em]
\tikzstyle{decision} = [diamond, aspect=2, draw, text centered, minimum height=2em]
\tikzstyle{data}=[trapezium, draw, text centered, trapezium left angle=60, trapezium right angle=120, minimum height=2em]
\tikzstyle{connector} = [line width=0.25mm,->]
%% [END] Tikz related stuff

%% [START] Fancy header related stuff
\usepackage{fancyhdr}
\pagestyle{fancy}
\setlength{\headheight}{15pt} % compensate fancyhdr style
\fancyhead{}
\fancyfoot{}
\fancyfoot[L]{\thepage}
\fancyfoot[R]{\textit{\vSubject - \vSubtitle}}
\renewcommand{\footrulewidth}{0.4pt}% default is 0pt, overline for footer
%% [END] Fancy header related stuff

%% [START] Custom tabular command related stuff
\usepackage{tabularx}
\newcommand{\details}[2]{
    #1 & #2  \\
}
%% [END] Custom tabular command related stuff

%% [START] Figure related stuff
\newcommand{\image}[3][1]{
    \begin{figure}[h]
        \centering
        \includegraphics[#1]{#2}
        \caption{#3}
        \label{#3}
    \end{figure}
}
%% [END] Figure related stuff

\begin{document}
\begin{titlepage}
    \centering
    \vfill
    {\bfseries\LARGE
        \vSubject\\
        \vskip0.25cm
        \vSubtitle
    }
    \vfill
    \includegraphics[width=6cm]{images/polinema-logo.png}
    \vfill
    {
        \textbf{Name}\\
        \vName\\
        \vskip0.5cm
        \textbf{NIM}\\
        \vNIM\\
        \vskip0.5cm
        \textbf{Class}\\
        \vClass\\
        \vskip0.5cm
        \textbf{Department}\\
        \vDepartment\\
        \vskip0.5cm
        \textbf{Study Program}\\
        \vStudyProgram
    }
\end{titlepage}

\section{Practice}
\begin{enumerate}
    \item {
        \begin{minted}[autogobble,fontsize=\small]{java}
            public class MyMultiplication {
                void multiplication(int a, int b) {
                    System.out.println(a * b);
                }

                void multiplication(int a, int b, int c) {
                    System.out.println(a * b * c);
                }

                public static void main(String[] args) {
                    MyMultiplication obj = new MyMultiplication();
                    obj.multiplication(25, 43);
                    obj.multiplication(34, 23, 56);
                }])
            }
        \end{minted}

        From the source code above, which one is the overloading?

        The overloading can be seen at the second method where we declare a method with the same name but with different number of parameters.
    }
    \item {
        If there are any overloading, how many different parameters are there?

        There are two different parameters. The first method having two parameters and the second method having three parameters.
    }
    \item {
        \begin{minted}[autogobble,fontsize=\small]{java}
            public class MyMultiplication {
                void multiplication(int a, int b) {
                    System.out.println(a * b);
                }

                void multiplication(float a, float b) {
                    System.out.println(a * b);
                }

                public static void main(String[] args) {
                    MyMultiplication obj = new MyMultiplication();
                    obj.multiplication(25, 43);
                    obj.multiplication(34.56, 23.7);
                }
            }
        \end{minted}

        From the source code above, which one is the overloading?

        The overloading can be seen at the second method where we declare a method with the same name but with different data type of parameters.
    }
    \item {
        If there are any overloading, how many different parameters are there?

        There are two different parameters. The first method having two integer parameters and the second method having two float parameters.
    }
    \item {
        \begin{minted}[autogobble,fontsize=\small]{java}
            class Fish {
                public void swim() {
                    System.out.println("Fish can swim");
                }
            }

            class Piranha extends Fish {
                public void swim() {
                    System.out.println("Piranha can eat meat");
                }
            }

            public class Main {
                public static void main(String[] args) {
                    Fish fish = new Fish();
                    Piranha piranha = new Piranha();
                    fish.swim();
                    piranha.swim();
                }
            }
        \end{minted}

        From the source code above, which one is the overriding?

        We override on the child class which is the \texttt{Piranha} class.
    }
    \item {
        Explain if the above souce code contains overriding!

        The above source code contains overriding because we override the \texttt{swim} method from the parent class which is the \texttt{Fish} class
        gets overriden by the child class which is the \texttt{Piranha} class.
    }
\end{enumerate}

\pagebreak

\section{Task}
\begin{enumerate}
    \item {
        \begin{minted}[autogobble,fontsize=\small]{java}
            public class Triangle {
                int angle;

                public Triangle(int angle) {
                    this.angle = angle;
                }

                public int angleTotal(int angle) {
                    return 180 - angle;
                }

                public int angleTotal(int angleA, int angleB) {
                    return 180 - (angleA + angleB);
                }

                public int parameter(int sideA, int sideB, int sideC) {
                    return sideA + sideB + sideC;
                }

                public double parameter(int sideA, int sideB) {
                    int sideC = (int) Math.sqrt(Math.pow(sideA, 2) + Math.pow(sideB, 2));
                    return sideA + sideB + sideC;
                }
            }
        \end{minted}
    }
    \item {
        \begin{minted}[autogobble,fontsize=\small]{java}
            public class Human {
                public void breath() {
                    System.out.println("Human can breath");
                }

                public void eat() {
                    System.out.println("Human can eat");
                }
            }

            public class Student extends Human {
                public void sleep() {
                    System.out.println("Student can sleep");
                }
            }

            public class Lecturer extends Human {
                public void workOvertime() {
                    System.out.println("Lecturer can work overtime");
                }
            }
        \end{minted}
    }
\end{enumerate}

\end{document}

