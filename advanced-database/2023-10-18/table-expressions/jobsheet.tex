\documentclass[12pt,titlepage]{article}
\usepackage[margin=1.25in]{geometry}
\usepackage{graphicx,amsmath,minted}

%% Variables definition
\newcommand{\vSubject}{Advanced Database}
\newcommand{\vSubtitle}{Table Expressions}
\newcommand{\vName}{Dicha Zelianivan Arkana}
\newcommand{\vNIM}{2241720002}
\newcommand{\vClass}{2i}
\newcommand{\vDepartment}{Information Technology}
\newcommand{\vStudyProgram}{D4 Informatics Engineering}

%% [START] Tikz related stuff
\usepackage{tikz}
\usetikzlibrary{svg.path,calc,shapes.geometric,shapes.misc}
\tikzstyle{terminator} = [rectangle, draw, text centered, rounded corners = 1em, minimum height=2em]
\tikzstyle{preparation} = [chamfered rectangle, chamfered rectangle sep=0.75em, draw, text centered, minimum height = 2em]
\tikzstyle{process} = [rectangle, draw, text centered, minimum height=2em]
\tikzstyle{decision} = [diamond, aspect=2, draw, text centered, minimum height=2em]
\tikzstyle{data}=[trapezium, draw, text centered, trapezium left angle=60, trapezium right angle=120, minimum height=2em]
\tikzstyle{connector} = [line width=0.25mm,->]
%% [END] Tikz related stuff

%% [START] Fancy header related stuff
\usepackage{fancyhdr}
\pagestyle{fancy}
\setlength{\headheight}{15pt} % compensate fancyhdr style
\fancyhead{}
\fancyfoot{}
\fancyfoot[L]{\thepage}
\fancyfoot[R]{\textit{\vSubject - \vSubtitle}}
\renewcommand{\footrulewidth}{0.4pt}% default is 0pt, overline for footer
%% [END] Fancy header related stuff

%% [START] Custom tabular command related stuff
\usepackage{tabularx}
\newcommand{\details}[2]{
    #1 & #2  \\
}
%% [END] Custom tabular command related stuff

%% [START] Figure related stuff
\newcommand{\image}[3][1]{
    \begin{figure}[h]
        \centering
        \includegraphics[#1]{#2}
        \caption{#3}
        \label{#3}
    \end{figure}
}
%% [END] Figure related stuff

\begin{document}
\begin{titlepage}
    \centering
    \vfill
    {\bfseries\LARGE
        \vSubject\\
        \vskip0.25cm
        \vSubtitle
    }
    \vfill
    \includegraphics[width=6cm]{images/polinema-logo.png}
    \vfill
    {
        \textbf{Name}\\
        \vName\\
        \vskip0.5cm
        \textbf{NIM}\\
        \vNIM\\
        \vskip0.5cm
        \textbf{Class}\\
        \vClass\\
        \vskip0.5cm
        \textbf{Department}\\
        \vDepartment\\
        \vskip0.5cm
        \textbf{Study Program}\\
        \vStudyProgram
    }
\end{titlepage}

\section{Practicum}
\begin{enumerate}
    \item {
        \begin{minted}[autogobble,fontsize=\small]{sql}
            SELECT
                productid,
                productname,
                supplierid,
                unitprice,
                discontinued
            FROM
                Production.Products
            WHERE
                categoryid = 1;
        \end{minted}
    }
    \item {
        \begin{minted}[autogobble,fontsize=\small]{sql}
            CREATE VIEW
                Production.ProductsBeverages
            AS SELECT
                productid,
                productname,
                supplierid,
                unitprice,
                discontinued
            FROM
                Production.Products
            WHERE
                categoryid = 1;
        \end{minted}
    }
    \item {
        \begin{minted}[autogobble,fontsize=\small]{sql}
            SELECT
                productid,
                productname
            FROM
                Production.ProductsBeverages
            WHERE
                supplierid = 1;
        \end{minted}
    }
    \pagebreak
    \item {
        Muncul pesan error seperti dibawah ini
        \begin{minted}[autogobble,fontsize=\footnotesize]{sql}
            The ORDER BY clause is invalid in views, inline functions, derived tables, subqueries, and common table expressions, 
            unless TOP, OFFSET or FOR XML is also specified.
        \end{minted}
        Hal ini dikarenakan kita tidak dapaet menggunakan klausa \texttt{ORDER BY} tanpa
        menggunakan klausa \texttt{TOP}, \texttt{OFFSET}, atau \texttt{FOR XML}.
        \begin{minted}[autogobble,fontsize=\small]{sql}
            ALTER VIEW
                Production.ProductsBeverages
            AS SELECT TOP(100) PERCENT
                productid,
                productname,
                supplierid,
                unitprice,
                discontinued
            FROM Production.Products
            WHERE categoryid = 1
            ORDER BY productname;
        \end{minted}
    }
    \item {
        Tidak, data tidak akan urut apabila kita tidak menggunakan klausa \texttt{ORDER BY}
    }
    \item {
        Muncul pesan error
        \begin{minted}[autogobble,fontsize=\footnotesize]{sql}
            Create View or Function failed because no column name was specified for column 6.
        \end{minted}
        Hal ini dikarenakan pada penggunaan klausa \texttt{CASE} kita tidak memberikan \textit{alias} pada hasilnya.
    }
    \item {
        \begin{minted}[autogobble,fontsize=\small]{sql}
            ALTER VIEW Production.ProductsBeverages AS
            SELECT
                productid,
                productname,
                supplierid,
                unitprice,
                discontinued,
                CASE
                    WHEN unitprice > 100. THEN N'high'
                    ELSE N'normal'
                END AS pricetype
            FROM Production.Products
            WHERE categoryid = 1;
        \end{minted}
    }
    \pagebreak
    \item {
        \begin{minted}[autogobble,fontsize=\small]{sql}
            SELECT
                p.productid, p.productname
            FROM
                (
                    SELECT
                        productid, productname, supplierid, unitprice, discontinued,
                        CASE
                            WHEN unitprice > 100. THEN N'high'
                            ELSE N'normal'
                        END AS pricetype
                    FROM Production.Products
                    WHERE categoryid = 1
                ) AS p
            WHERE p.pricetype = N'high';
        \end{minted}
    }
    \item {
        \begin{minted}[autogobble,fontsize=\small]{sql}
            SELECT
                c.custid,
                SUM(c.totalsalesamountperorder) AS totalsalesamount,
                AVG(c.totalsalesamountperorder) AS avgsalesamount
            FROM
                (
                    SELECT
                        o.custid,
                        o.orderid,
                        SUM(d.unitprice * d.qty) AS totalsalesamountperorder
                    FROM
                        Sales.Orders AS o
                    INNER JOIN
                        Sales.OrderDetails d ON d.orderid = o.orderid
                    GROUP BY
                        o.custid, o.orderid
                ) AS c
            GROUP BY c.custid;
        \end{minted}
    }
    \pagebreak
    \item {
        \begin{minted}[autogobble,fontsize=\small]{sql}
            SELECT
                cy.orderyear,
                cy.totalsalesamount AS curtotalsales,
                py.totalsalesamount AS prevtotalsales,
                ((cy.totalsalesamount - py.totalsalesamount) 
                    / py.totalsalesamount 
                    * 100.) AS percentgrowth
            FROM
                (
                    SELECT
                        YEAR(orderdate) AS orderyear,
                        SUM(val) AS totalsalesamount
                    FROM Sales.OrderValues
                    GROUP BY YEAR(orderdate)
                ) AS cy
            LEFT OUTER JOIN
                (
                    SELECT
                        YEAR(orderdate) AS orderyear,
                        SUM(val) AS totalsalesamount
                    FROM Sales.OrderValues
                    GROUP BY YEAR(orderdate)
                ) AS py
                ON cy.orderyear = py.orderyear + 1
            ORDER BY cy.orderyear;
        \end{minted}
    }
    \item {
        \begin{minted}[autogobble,fontsize=\small]{sql}
            WITH ProductsBeverages AS
                (
                    SELECT
                        productid,
                        productname,
                        supplierid,
                        unitprice,
                        discontinued,
                        CASE
                            WHEN unitprice > 100. THEN N'high'
                            ELSE N'normal'
                        END AS pricetype
                    FROM Production.Products
                    WHERE categoryid = 1
                )
            SELECT
                productid,
                productname
            FROM ProductsBeverages
            WHERE pricetype = N'high';
        \end{minted}
    }
    \item {
        \begin{minted}[autogobble,fontsize=\small]{sql}
            WITH c2008 (custid, salesamt2008) AS
                (
                    SELECT
                        custid,
                        SUM(val)
                    FROM Sales.OrderValues
                    WHERE YEAR(orderdate) = 2008
                    GROUP BY custid
                )
            SELECT
                c.custid,
                c.contactname,
                c2008.salesamt2008
            FROM Sales.Customers AS c
            LEFT OUTER JOIN
                c2008 ON c.custid = c2008.custid;
        \end{minted}
    }
    \item {
        \begin{minted}[autogobble,fontsize=\small]{sql}
            WITH 
                c2008 (custid, salesamt2008) AS
                    (
                        SELECT
                        custid, SUM(val)
                        FROM Sales.OrderValues
                        WHERE YEAR(orderdate) = 2008
                        GROUP BY custid
                    ),
                c2007 (custid, salesamt2007) AS
                    (
                        SELECT
                        custid, SUM(val)
                        FROM Sales.OrderValues
                        WHERE YEAR(orderdate) = 2007
                        GROUP BY custid
                    )
            SELECT
                c.custid, 
                c.contactname,
                c2008.salesamt2008,
                c2007.salesamt2007,
                COALESCE(
                    (c2008.salesamt2008 - c2007.salesamt2007) / c2007.salesamt2007 * 100.,
                    0
                )
            AS percentgrowth
            FROM Sales.Customers AS c
            LEFT OUTER JOIN
                c2008 ON c.custid = c2008.custid
            LEFT OUTER JOIN
                c2007 ON c.custid = c2007.custid
            ORDER BY percentgrowth DESC;
        \end{minted}
    }
    \item {
        \begin{minted}[autogobble,fontsize=\small]{sql}
            SELECT
                custid,
                SUM(val) AS totalsalesamount
            FROM 
                Sales.OrderValues
            WHERE
                YEAR(orderdate) = 2007
            GROUP BY
                custid;
        \end{minted}
    }
    \item {
        \begin{minted}[autogobble,fontsize=\small]{sql}
            CREATE FUNCTION dbo.fnGetSalesByCustomer
                (@orderyear AS INT) RETURNS TABLE
            AS RETURN
            SELECT
                custid,
                SUM(val) AS totalsalesamount
            FROM 
                Sales.OrderValues
            WHERE
                YEAR(orderdate) = 2007
            GROUP BY
                custid;
        \end{minted}
    }
    \item {
        \begin{minted}[autogobble,fontsize=\small]{sql}
            CREATE FUNCTION dbo.fnGetSalesByCustomer
                (@orderyear AS INT) RETURNS TABLE
            AS RETURN
            SELECT
                custid,
                SUM(val) AS totalsalesamount
            FROM
                Sales.OrderValues
            WHERE
                YEAR(orderdate) = @orderyear
            GROUP BY
                custid;
        \end{minted}
    }
    \pagebreak
    \item {
        \begin{minted}[autogobble,fontsize=\small]{sql}
            SELECT
                custid, 
                totalsalesamount
            FROM
                dbo.fnGetSalesByCustomer(2007);
        \end{minted}
    }
    \item {
        \begin{minted}[autogobble,fontsize=\small]{sql}
            SELECT TOP(3)
                d.productid,
                MAX(p.productname) AS productname,
                SUM(d.qty * d.unitprice) AS totalsalesamount
            FROM Sales.Orders AS o
            INNER JOIN Sales.OrderDetails AS d ON d.orderid = o.orderid
            INNER JOIN Production.Products AS p ON p.productid = d.productid
            WHERE custid = 1
            GROUP BY d.productid
            ORDER BY totalsalesamount DESC;
        \end{minted}
    }
    \item {
        \begin{minted}[autogobble,fontsize=\small]{sql}
            SELECT TOP(3)
                d.productid,
                MAX(p.productname) AS productname,
                SUM(d.qty * d.unitprice) AS totalsalesamount
            FROM 
                Sales.Orders AS o
            INNER JOIN
                Sales.OrderDetails AS d ON d.orderid = o.orderid
            INNER JOIN
                Production.Products AS p ON p.productid = d.productid
            WHERE custid = 1
            GROUP BY d.productid
            ORDER BY totalsalesamount DESC;
        \end{minted}
    }
    \pagebreak
    \item {
        \begin{minted}[autogobble,fontsize=\small]{sql}
            CREATE FUNCTION dbo.fnGetTop3ProductsForCustomer
                (@custid AS INT) RETURNS TABLE
            AS RETURN
            SELECT TOP(3)
                d.productid,
                MAX(p.productname) AS productname,
                SUM(d.qty * d.unitprice) AS totalsalesamount
            FROM
                Sales.Orders AS o
            INNER JOIN
                Sales.OrderDetails AS d ON d.orderid = o.orderid
            INNER JOIN
                Production.Products AS p ON p.productid = d.productid
            WHERE custid = @custid
            GROUP BY d.productid
            ORDER BY totalsalesamount DESC;


            SELECT
                p.productid,
                p.productname,
                p.totalsalesamount
            FROM 
                dbo.fnGetTop3ProductsForCustomer(1) AS p;
        \end{minted}
    }
\end{enumerate}

\end{document}

