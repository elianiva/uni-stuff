\documentclass[12pt,titlepage]{article}
\usepackage[margin=1.25in]{geometry}
\usepackage{graphicx,amsmath,minted}

%% Variables definition
\newcommand{\vSubject}{Advanced Database}
\newcommand{\vSubtitle}{Data Type}
\newcommand{\vName}{Dicha Zelianivan Arkana}
\newcommand{\vNIM}{2241720002}
\newcommand{\vClass}{2i}
\newcommand{\vDepartment}{Information Technology}
\newcommand{\vStudyProgram}{D4 Informatics Engineering}

%% [START] Tikz related stuff
\usepackage{tikz}
\usetikzlibrary{svg.path,calc,shapes.geometric,shapes.misc}
\tikzstyle{terminator} = [rectangle, draw, text centered, rounded corners = 1em, minimum height=2em]
\tikzstyle{preparation} = [chamfered rectangle, chamfered rectangle sep=0.75em, draw, text centered, minimum height = 2em]
\tikzstyle{process} = [rectangle, draw, text centered, minimum height=2em]
\tikzstyle{decision} = [diamond, aspect=2, draw, text centered, minimum height=2em]
\tikzstyle{data}=[trapezium, draw, text centered, trapezium left angle=60, trapezium right angle=120, minimum height=2em]
\tikzstyle{connector} = [line width=0.25mm,->]
%% [END] Tikz related stuff

%% [START] Fancy header related stuff
\usepackage{fancyhdr}
\pagestyle{fancy}
\setlength{\headheight}{15pt} % compensate fancyhdr style
\fancyhead{}
\fancyfoot{}
\fancyfoot[L]{\thepage}
\fancyfoot[R]{\textit{\vSubject - \vSubtitle}}
\renewcommand{\footrulewidth}{0.4pt}% default is 0pt, overline for footer
%% [END] Fancy header related stuff

%% [START] Custom tabular command related stuff
\usepackage{tabularx}
\newcommand{\details}[2]{
    #1 & #2  \\
}
%% [END] Custom tabular command related stuff

%% [START] Figure related stuff
\newcommand{\image}[3][1]{
    \begin{figure}[h]
        \centering
        \includegraphics[#1]{#2}
        \caption{#3}
        \label{#3}
    \end{figure}
}
%% [END] Figure related stuff

\begin{document}
\begin{titlepage}
    \centering
    \vfill
    {\bfseries\LARGE
        \vSubject\\
        \vskip0.25cm
        \vSubtitle
    }
    \vfill
    \includegraphics[width=6cm]{images/polinema-logo.png}
    \vfill
    {
        \textbf{Name}\\
        \vName\\
        \vskip0.5cm
        \textbf{NIM}\\
        \vNIM\\
        \vskip0.5cm
        \textbf{Class}\\
        \vClass\\
        \vskip0.5cm
        \textbf{Department}\\
        \vDepartment\\
        \vskip0.5cm
        \textbf{Study Program}\\
        \vStudyProgram
    }
\end{titlepage}

\section{Practicum}
\begin{enumerate}
    \item {
        \begin{minted}[autogobble,fontsize=\small]{sql}
            SELECT
                GETDATE() AS currentdatetime,
                FORMAT(GETDATE(), 'HH:mm:ss') AS currentdate,
                YEAR(GETDATE()) AS currentyear,
                MONTH(GETDATE()) AS currentmonth,
                DAY(GETDATE()) AS currentday,
                DATEPART(WEEK, GETDATE()) AS currentweeknumber,
                DATENAME(MONTH, GETDATE()) AS currentmonthname;
        \end{minted}
    }
    \item {
        Tidak, karena tidak diperbolehkan adanya duplikat dalam penamaan alias
    }
    \item {
        \begin{minted}[autogobble,fontsize=\small]{sql}
            SELECT CAST('1945-08-17' AS DATE) AS somedate;
            SELECT CONVERT(DATE, '1945-08-17', 23) AS somedate;
            SELECT DATEFROMPARTS(1945, 8, 17) AS somedate;
        \end{minted}
    }
    \item {
        \begin{minted}[autogobble,fontsize=\small]{sql}
            SELECT
                DATEADD(MONTH, 3, GETDATE()) AS threemonths,
                DATEDIFF(DAY, GETDATE(), DATEADD(MONTH, 3, GETDATE())) AS diffdays,
                DATEDIFF(WEEK, '1945-08-17', '2018-10-01') AS diffweeks,
                DATEADD(DAY, 1 - DAY(GETDATE()), GETDATE()) AS firstday;
        \end{minted}
    }
    \item {
        \begin{minted}[autogobble,fontsize=\small]{sql}
            SELECT isitdate,
                CASE
                    WHEN ISDATE(isitdate) = 1 THEN CONVERT(date, isitdate)
                    ELSE NULL
                END AS converteddate
            FROM Sales.Somedates;
        \end{minted}
    }
    \item {
        \texttt{SYSDATETIME} mengembalikan nilai bertipe \texttt{DATETIME2(7)}
        \\ sedangkan \texttt{CURRENT\_TIMESTAMP} mengembalikan hasil bertipe \texttt{DATETIME}
    }
    \item {
        \begin{minted}[autogobble,fontsize=\small]{sql}
            SELECT DISTINCT custid
            FROM Sales.Orders
            WHERE YEAR(orderdate) = 2008 AND MONTH(orderdate) = 2;
        \end{minted}
    }
    \item {
        \begin{minted}[autogobble,fontsize=\footnotesize]{sql}
            SELECT 
                GETDATE() AS currentdate,
                DATEADD(month, DATEDIFF(month, 0, GETDATE()), 0) AS firstofmonth,
                DATEADD(day, -1, DATEADD(month, DATEDIFF(month, 0, GETDATE()) + 1, 0)) AS endofmonth;
        \end{minted}
    }
    \pagebreak
    \item {
        \begin{minted}[autogobble,fontsize=\small]{sql}
            SELECT
                orderid,
                custid,
                orderdate
            FROM 
                Sales.Orders
            WHERE
                orderdate >= DATEADD(day, -5, DATEADD(month, DATEDIFF(month, 0, GETDATE()), 0))
                AND orderdate < DATEADD(month, DATEDIFF(month, 0, GETDATE()) + 1, 0);
        \end{minted}
    }
    \item {
        \begin{minted}[autogobble,fontsize=\small]{sql}
            SELECT
                orderid,
                custid,
                orderdate
            FROM 
                Sales.Orders
            WHERE
                DAY(orderdate) > DAY(EOMONTH(orderdate, 0)) - 5
        \end{minted}
    }
    \item {
        \begin{minted}[autogobble,fontsize=\small]{sql}
            SELECT DISTINCT
                od.productid
            FROM
                Sales.Orders o
            JOIN
                Sales.OrderDetails od ON o.orderid = od.orderid
            WHERE
                YEAR(orderdate) = 2007
                AND DATEPART(month, orderdate) <= 10
            ORDER BY od.productid;
        \end{minted}
    }
    \item {
        \begin{minted}[autogobble,fontsize=\small]{sql}
            SELECT
                CONCAT(contactname, ' (city: ', city, ')') AS contactwithcity
            FROM
                Sales.Customers;
        \end{minted}
    }
    \pagebreak
    \item {
        \begin{minted}[autogobble,fontsize=\small]{sql}
            SELECT
                CONCAT(
                    contactname,
                    ' (city: ',
                    city, 
                    ', region: ',
                    normalisedregion,
                    ')'
                ) AS contactwithcity
            FROM
                (
                    SELECT
                        normalisedregion = CASE
                            WHEN region IS NULL THEN ''
                            ELSE region
                        END,
                        city,
                        contactname
                    FROM
                        Sales.Customers
                ) AS c;
        \end{minted}
    }
    \item {
        \begin{minted}[autogobble,fontsize=\small]{sql}
            SELECT
                contactname,
                contacttitle
            FROM
                Sales.Customers
            WHERE
                ASCII(LEFT(contactname, 1)) >= 65
                AND ASCII(LEFT(contactname, 1)) <= 71
            ORDER BY
                contactname;
        \end{minted}
    }
    \item {
        \begin{minted}[autogobble,fontsize=\small]{sql}
            SELECT
                contactname,
                LEFT(contactname, CHARINDEX(', ', contactname) - 1) AS lastname
            FROM
                Sales.Customers;
        \end{minted}
    }
    \item {
        \begin{minted}[autogobble,fontsize=\small]{sql}
            SELECT
                REPLACE(contactname, ', ', ' ') as newcontactname,
                RIGHT(contactname, CHARINDEX(' ', REVERSE(contactname))) as firstname
            FROM
                Sales.Customers;
        \end{minted}
    }
    \item {
        \begin{minted}[autogobble,fontsize=\small]{sql}
            SELECT
                custid,
                CONCAT('C', FORMAT(custid, '00000')) as newcustid
            FROM
                Sales.Customers;
        \end{minted}
    }
    \item {
        \begin{minted}[autogobble,fontsize=\small]{sql}
            SELECT
                contactname,
                LEN(contactname) - LEN(
                    REPLACE(contactname COLLATE SQL_Latin1_General_Cp1_CI_AI, 'a', '')
                ) as numberofa
            FROM
                Sales.Customers
            ORDER BY
                numberofa DESC;
        \end{minted}
    }
\end{enumerate}

\end{document}

