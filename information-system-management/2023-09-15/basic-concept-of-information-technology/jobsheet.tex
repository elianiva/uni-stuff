\documentclass[12pt,titlepage]{article}
\usepackage[margin=1.25in]{geometry}
\usepackage{graphicx,amsmath,blindtext}

%% Variables definition
\newcommand{\vSubject}{Information System Management}
\newcommand{\vSubtitle}{Basic Concept of Information Technology}
\newcommand{\vName}{Dicha Zelianivan Arkana}
\newcommand{\vNIM}{2241720002}
\newcommand{\vClass}{2i}
\newcommand{\vDepartment}{Information Technology}
\newcommand{\vStudyProgram}{D4 Informatics Engineering}

%% [START] Tikz related stuff
\usepackage{tikz}
\usetikzlibrary{svg.path,calc,shapes.geometric,shapes.misc}
\tikzstyle{terminator} = [rectangle, draw, text centered, rounded corners = 1em, minimum height=2em]
\tikzstyle{preparation} = [chamfered rectangle, chamfered rectangle sep=0.75em, draw, text centered, minimum height = 2em]
\tikzstyle{process} = [rectangle, draw, text centered, minimum height=2em]
\tikzstyle{decision} = [diamond, aspect=2, draw, text centered, minimum height=2em]
\tikzstyle{data}=[trapezium, draw, text centered, trapezium left angle=60, trapezium right angle=120, minimum height=2em]
\tikzstyle{connector} = [line width=0.25mm,->]
%% [END] Tikz related stuff

%% [START] Fancy header related stuff
\usepackage{fancyhdr}
\pagestyle{fancy}
\setlength{\headheight}{15pt} % compensate fancyhdr style
\fancyhead{}
\fancyfoot{}
\fancyfoot[L]{\thepage}
\fancyfoot[R]{\textit{\vSubject - \vSubtitle}}
\renewcommand{\footrulewidth}{0.4pt}% default is 0pt, overline for footer
%% [END] Fancy header related stuff

%% [START] Custom tabular command related stuff
\usepackage{tabularx}
\newcommand{\details}[2]{
    #1 & #2  \\
}
%% [END] Custom tabular command related stuff

%% [START] Figure related stuff
\newcommand{\image}[3][1]{
    \begin{figure}[h]
        \centering
        \includegraphics[#1]{#2}
        \caption{#3}
        \label{#3}
    \end{figure}
}
%% [END] Figure related stuff

\begin{document}
\begin{titlepage}
    \centering
    \vfill
    {\bfseries\LARGE
        \vSubject\\
        \vskip0.25cm
        \vSubtitle
    }
    \vfill
    \includegraphics[width=6cm]{images/polinema-logo.png}
    \vfill
    {
        \textbf{Name}\\
        \vName\\
        \vskip0.5cm
        \textbf{NIM}\\
        \vNIM\\
        \vskip0.5cm
        \textbf{Class}\\
        \vClass\\
        \vskip0.5cm
        \textbf{Department}\\
        \vDepartment\\
        \vskip0.5cm
        \textbf{Study Program}\\
        \vStudyProgram
    }
\end{titlepage}

\tableofcontents

\pagebreak

\section{Basic Concept of Information Technology}
\subsection{Definition of Information Technology}
\textbf{Information Technology} has several definitions from the experts, but in short it can
be defined as a set of tools or technology that is used to process and distribute information.

\subsection{Understanding Information Technology Concepts}
\textbf{Information Technology} is a sub-system or part of a larger system called \textbf{Information System}.
More often than not, the term \textit{Technology} and \textit{Information System} are overlapping, but
the term \textit{Information Technology} is more focused on the technology itself rather than the larger system.

\subsection{Grouping of Information Technology}
\textbf{Information Technology} can be grouped into 5 categories:
\begin{enumerate}
    \item {
        \textbf{Input}

        Technology related to the input of data into the system, such as \textit{keyboard} and \textit{mouse}.
    }
    \item {
        \textbf{Processing}

        Technology related to the processing of data, such as \textit{CPU} and \textit{GPU}. This is where most
        of the processing happens. The data that has been inputted will be processed here.
    }
    \item {
        \textbf{Storage}

        Technology related to the storage of data. This is divided into 2 categories:
        \begin{enumerate}
            \item {
                \textbf{Internal Storage / Main Storage}

                Technology related to the storage of data that is currently being processed, such as \textit{RAM}, \textit{ROM}, etc.
                It's usually faster than \textit{External Storage}. It needs constant electricity to operate since it's a
                volatile storage.
            }
            \item {
                \textbf{External Storage / Secondary Storage}

                Technology related to the storage of data that is not currently being processed, such as \textit{HDD} and \textit{SSD}.
                It's usually slower than \textit{Internal Storage} and has a larger capacity. It also doesn't need constant electricity,
                meaning that it's not volatile.
            }
        \end{enumerate}
    }
    \item {
        \textbf{Output}

        Technology related to the output of data from the system, such as \textit{monitor} and \textit{printer}.
        The data that has been processed will be displayed or printed.
    }
    \item {
        \textbf{Software}

        Software is a set of instructions that is used to control the hardware. There are usually
        software for a specific task, such as \textit{Microsoft Word} for word processing, \textit{Microsoft Excel}
        for spreadsheet, etc.

        This is one of the most important part of the system because it acts as a bridge to connect the user
        to the hardware.
    }
\end{enumerate}

\subsection{Information Technology System Components}
\textbf{Information Technology System} consists of multiple components, both that can be seen and can't be seen.
The components that can be seen are called \textbf{Hardware}, while the components that can't be seen are called
\textbf{Software}. The network can also be considered as one of its component, but it's not always present.
The main component that controls the entire system is the user which can also be called as the \textbf{Brainware}.

\subsection{Information Technology Roles}
\textbf{Information Technology} has many roles in our life. It can be used to help us in our daily life, such as
\textit{online shopping}, \textit{online transportation}, \textit{online food delivery}, etc. It can also be used
to help us in our work, such as \textit{Microsoft Word} for word processing, \textit{Microsoft Excel} for spreadsheet,
etc.

In conclusion, Information Technology plays an important role in our daily life almost in every aspect.
We've been integrating it into our life so much that we can't live without it. It's also constantly evolving
to help us in our daily life.

\section{Information Technology Development}
\subsection{Information Technology Understanding}
As of now, Information Technology has been understood by many people as a daily necessity.
It's no longer a luxury item that only a few people can afford.

\subsection{Evolution of Technology}
Technology has been evolving rapidly since the beginning of time. It's constantly evolving to get better and better.
At this day and age, the economy revolve around information and technology.

\subsection{Advancement of Information Technology}
Information Technology has been advancing rapidly since it's first discovered. Even now it's still advancing
rapidly to provide us with better technology. For example, we no longer need to shop at the store to buy something,
we can just buy it online and it will be delivered to our house. We no longer need pen and paper to write something,
we can just use our phone or computer.

Globalisation plays a big role in the advancement of Information Technology.
It makes it easier for us to exchange information with other people around the world.
This come with its own risk, such as \textit{fake news}, \textit{hoax}, etc.

\subsection{Application of Information Technology}
Humans are social creatures, we need to interact with other people to survive.
Our basic necessity consists of food, water, and shelter. But now, we also need
information to survive. Information Technology helps us to get information from 
other people around the world.

\end{document}

