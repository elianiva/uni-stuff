\documentclass[12pt,titlepage]{article}
\usepackage[margin=1.25in]{geometry}
\usepackage{graphicx,amsmath}

%% Variables definition
\newcommand{\vSubject}{Information System Management}
\newcommand{\vSubtitle}{Chapter 4 and 5}
\newcommand{\vName}{Dicha Zelianivan Arkana}
\newcommand{\vNIM}{2241720002}
\newcommand{\vClass}{2i}
\newcommand{\vDepartment}{Information Technology}
\newcommand{\vStudyProgram}{D4 Informatics Engineering}

%% [START] Tikz related stuff
\usepackage{tikz}
\usetikzlibrary{svg.path,calc,shapes.geometric,shapes.misc}
\tikzstyle{terminator} = [rectangle, draw, text centered, rounded corners = 1em, minimum height=2em]
\tikzstyle{preparation} = [chamfered rectangle, chamfered rectangle sep=0.75em, draw, text centered, minimum height = 2em]
\tikzstyle{process} = [rectangle, draw, text centered, minimum height=2em]
\tikzstyle{decision} = [diamond, aspect=2, draw, text centered, minimum height=2em]
\tikzstyle{data}=[trapezium, draw, text centered, trapezium left angle=60, trapezium right angle=120, minimum height=2em]
\tikzstyle{connector} = [line width=0.25mm,->]
%% [END] Tikz related stuff

%% [START] Fancy header related stuff
\usepackage{fancyhdr}
\pagestyle{fancy}
\setlength{\headheight}{15pt} % compensate fancyhdr style
\fancyhead{}
\fancyfoot{}
\fancyfoot[L]{\thepage}
\fancyfoot[R]{\textit{\vSubject - \vSubtitle}}
\renewcommand{\footrulewidth}{0.4pt}% default is 0pt, overline for footer
%% [END] Fancy header related stuff

%% [START] Custom tabular command related stuff
\usepackage{tabularx}
\newcommand{\details}[2]{
    #1 & #2  \\
}
%% [END] Custom tabular command related stuff

%% [START] Figure related stuff
\newcommand{\image}[3][1]{
    \begin{figure}[h]
        \centering
        \includegraphics[#1]{#2}
        \caption{#3}
        \label{#3}
    \end{figure}
}
%% [END] Figure related stuff

\begin{document}
\begin{titlepage}
    \centering
    \vfill
    {\bfseries\LARGE
        \vSubject\\
        \vskip0.25cm
        \vSubtitle
    }
    \vfill
    \includegraphics[width=6cm]{images/polinema-logo.png}
    \vfill
    {
        \textbf{Name}\\
        \vName\\
        \vskip0.5cm
        \textbf{NIM}\\
        \vNIM\\
        \vskip0.5cm
        \textbf{Class}\\
        \vClass\\
        \vskip0.5cm
        \textbf{Department}\\
        \vDepartment\\
        \vskip0.5cm
        \textbf{Study Program}\\
        \vStudyProgram
    }
\end{titlepage}

\section{Facts, Data, and Information}
\subsection{Definitions}
\begin{itemize}
    \item \textbf{Fact}: a statement about something that can be verified as being true or false.
    \item \textbf{Data}: a collection of facts organized in such a way that they have additional value beyond the value of the facts themselves.
    \item \textbf{Information}: data that has been organized and processed so that it is meaningful to the person who receives it.
    \item \textbf{Concepts}: ideas, opinions, and knowledge.
    \item \textbf{Principle}: a rule or law that is used as the basis for making judgments or decisions.
    \item \textbf{Law}: a specific principle that is universally accepted as true and that can be used as a basis for reasoning or conduct.
    \item \textbf{Theory}: a system of ideas that explains many related observations and is supported by a large body of evidence acquired through scientific investigation.
\end{itemize}

\subsection{Data Classification}
\subsubsection{Characteristics}
Data can be classified as two types based on its characteristics:
\begin{itemize}
    \item \textbf{Quantitative data}: data that can be measured and expressed numerically.
    \item \textbf{Qualitative data}: data that is descriptive and often based on observation, interviews, or the experiences of people.
\end{itemize}

\subsubsection{Source}
Data can be classified as two types based on its source:
\begin{itemize}
    \item \textbf{Primary data}: data that is collected directly from the source.
    \item \textbf{Secondary data}: data that is collected from a source that has already collected and formatted the data.
\end{itemize}

\subsection{Information}
\subsubsection{Information Characteristics}
There are eleven characteristics of information according to Nicholas (Ishak, 2006: 94):
\begin{enumerate}
    \item Subject
    \item Function
    \item Nature
    \item Intellectual Level
    \item View Point
    \item Quantity
    \item Quality
    \item Date
    \item Speed of Delivery
    \item Place or Origin
    \item Processing and Packaging
\end{enumerate}

\subsubsection{Types of Information}
There are three types of information:
\begin{itemize}
    \item \textbf{Scorekeeping Information}: accumulation of data to answer questions
    \item \textbf{Attention-directing Information}: data that is used to direct the attention of the decision maker to a problem or opportunity
    \item \textbf{Problem-solving Information}: data that is used to solve a problem
\end{itemize}

\subsubsection{Characteristics}
There are four characteristics of information:
\begin{itemize}
    \item Information must be pertinent
    \item Information must be accurate
    \item Information must be timely
    \item Information must be relevant
\end{itemize}

\section{Concepts of Information System Management}
\subsection{Definitions}
\begin{itemize}
    \item \textbf{Information System Management}: a system that was made to provide information that supports in the management activity in an organisation
\end{itemize}

\subsection{Parts of Information System Management}
\begin{itemize}
    \item Accounting Information System
    \item Marketing Information System
    \item Inventory Management Information System
    \item Personnal Information System
    \item Distribution Information System
    \item Purchasing Information System
    \item Treasury Information System
    \item Credit Analysis Information System
\end{itemize}

\subsection{Information System Management Functions}
\begin{itemize}
    \item Improving operational efficiency
    \item Introducing innovation in the business process
    \item Creating a strategic source of information
\end{itemize}

\subsection{Information System Management Components}
According to its functions, Information System Management consists of five components:
\begin{itemize}
    \item Administration and Operational System
    \item Reporting System
    \item Database System
    \item Searching System
    \item Data Management System
\end{itemize}

\end{document}

