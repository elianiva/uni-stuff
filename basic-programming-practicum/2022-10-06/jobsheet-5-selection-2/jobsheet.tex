\documentclass[12pt,titlepage]{article}
\usepackage[margin=1.25in]{geometry}
\usepackage{graphicx,amsmath,blindtext,minted}

%% Variables definition
\newcommand{\vSubject}{Basic Programming Practicum}
\newcommand{\vSubtitle}{Jobsheet 5 Selection 2}
\newcommand{\vName}{Dicha Zelianivan Arkana}
\newcommand{\vNIM}{2241720002}
\newcommand{\vClass}{1i}
\newcommand{\vDepartment}{Information Technology}
\newcommand{\vStudyProgram}{D4 Informatics Engineering}

%% [START] Tikz related stuff
\usepackage{tikz}
\usetikzlibrary{svg.path,calc,shapes.geometric,shapes.misc}
\tikzstyle{terminator} = [rectangle, draw, text centered, rounded corners = 1em, minimum height=2em]
\tikzstyle{preparation} = [chamfered rectangle, chamfered rectangle sep=0.75em, draw, text centered, minimum height = 2em]
\tikzstyle{process} = [rectangle, draw, text centered, minimum height=2em]
\tikzstyle{decision} = [diamond, aspect=2, draw, text centered, minimum height=2em]
\tikzstyle{data}=[trapezium, draw, text centered, trapezium left angle=60, trapezium right angle=120, minimum height=2em]
\tikzstyle{connector} = [line width=0.25mm,->]
%% [END] Tikz related stuff

%% [START] Fancy header related stuff
\usepackage{fancyhdr}
\pagestyle{fancy}
\setlength{\headheight}{15pt} % compensate fancyhdr style
\fancyhead{}
\fancyfoot{}
\fancyfoot[L]{\thepage}
\fancyfoot[R]{\textit{\vSubject - \vSubtitle}}
\renewcommand{\footrulewidth}{0.4pt}% default is 0pt, overline for footer
%% [END] Fancy header related stuff

%% [START] Custom tabular command related stuff
\usepackage{tabularx}
\newcommand{\details}[2]{
    #1 & #2  \\
}
%% [END] Custom tabular command related stuff

%% [START] Figure related stuff
\newcommand{\image}[3][1]{
    \begin{figure}[h]
        \centering
        \includegraphics[#1]{#2}
        \caption{#3}
        \label{#3}
    \end{figure}
}
%% [END] Figure related stuff

\begin{document}
\begin{titlepage}
    \centering
    \vfill
    {\bfseries\LARGE
        \vSubject\\
        \vskip0.25cm
        \vSubtitle
    }
    \vfill
    \includegraphics[width=6cm]{images/polinema-logo.png}
    \vfill
    {
        \textbf{Name}\\
        \vName\\
        \vskip0.5cm
        \textbf{NIM}\\
        \vNIM\\
        \vskip0.5cm
        \textbf{Class}\\
        \vClass\\
        \vskip0.5cm
        \textbf{Department}\\
        \vDepartment\\
        \vskip0.5cm
        \textbf{Study Program}\\
        \vStudyProgram
    }
\end{titlepage}

\tableofcontents

\pagebreak

\section{Laboratory}
\subsection{Experiment 1}

\begin{enumerate}
    \item Open a text editor. Create a new file, name it \textbf{Nested1.java}
    \item Write the basic structure of the Java programming language which contains the \texttt{main()} function
    \item Add the \texttt{Scanner} library.
    \item Make a \texttt{Scanner} declaration with the name sc
    \item Create an \texttt{int} variable with the name \texttt{value}
    \item {
        Write down the syntax for entering the value from keyboard

        \begin{minted}[autogobble,fontsize=\small]{java}
            System.out.print("Enter a value (0 - 100): ");
            value = sc.nextInt();
        \end{minted}
    }
    \item {
        Create a nested selection structure. The first check is used to ensure that the value
        entered is in the range 0 - 100. If the value is in the range 0 - 100, then a student
        graduation status will be checked, i.e. if the value is between 90 - 100 then the value is
        A, if the value is between 80 - 89 then the value is B, if the value is between 60 - 79 then
        the value is C, if the value is between 50 - 59 then the value is D, and if the value is
        between 0 - 49 then the value is E. Whereas if the value is outside the range 0 - 100 , then
        displayed information stating that the value entered is invalid.

        \begin{minted}[autogobble,fontsize=\small]{java}
            if (value >= 0 && value <= 100) {
                if (value >= 90 && value <= 100) {
                    System.out.println("Grade A, EXCELLENT!");
                } else if (value >= 80 && value <= 89) {
                    System.out.println("Grade B, keep up your achievements!");
                } else if (value >= 60 && value <= 79) {
                    System.out.println("Grade C, increase your achievements!");
                } else if (value >= 50 && value <= 59) {
                    System.out.println("Grade D, improve your study!");
                } else {
                    System.out.println("Grade E, you don't pass!");
                }
            } else {
                System.out.println("The value you entered is invalid");
            }
        \end{minted}
    }
    \pagebreak
    \item {
        Compile and run the program. Observe the results!

        
    }
\end{enumerate}

\end{document}

