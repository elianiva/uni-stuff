\documentclass[12pt,titlepage]{article}
\usepackage[margin=1.25in]{geometry}
\usepackage{graphicx,amsmath,tabularx}

%% Variables definition
\newcommand{\vSubject}{Artificial Intelligence}
\newcommand{\vSubtitle}{Midterm Exam}
\newcommand{\vName}{Dicha Zelianivan Arkana}
\newcommand{\vNIM}{2241720002}
\newcommand{\vClass}{2i}
\newcommand{\vDepartment}{Information Technology}
\newcommand{\vStudyProgram}{D4 Informatics Engineering}

%% [START] Tikz related stuff
\usepackage{tikz}
\usetikzlibrary{svg.path,calc,shapes.geometric,shapes.misc}
\tikzstyle{terminator} = [rectangle, draw, text centered, rounded corners = 1em, minimum height=2em]
\tikzstyle{preparation} = [chamfered rectangle, chamfered rectangle sep=0.75em, draw, text centered, minimum height = 2em]
\tikzstyle{process} = [rectangle, draw, text centered, minimum height=2em]
\tikzstyle{decision} = [diamond, aspect=2, draw, text centered, minimum height=2em]
\tikzstyle{data}=[trapezium, draw, text centered, trapezium left angle=60, trapezium right angle=120, minimum height=2em]
\tikzstyle{connector} = [line width=0.25mm,->]
%% [END] Tikz related stuff

%% [START] Fancy header related stuff
\usepackage{fancyhdr}
\pagestyle{fancy}
\setlength{\headheight}{15pt} % compensate fancyhdr style
\fancyhead{}
\fancyfoot{}
\fancyfoot[L]{\thepage}
\fancyfoot[R]{\textit{\vSubject - \vSubtitle}}
\renewcommand{\footrulewidth}{0.4pt}% default is 0pt, overline for footer
%% [END] Fancy header related stuff

%% [START] Custom tabular command related stuff
\usepackage{tabularx}
\newcommand{\details}[2]{
    #1 & #2  \\
}
%% [END] Custom tabular command related stuff

%% [START] Figure related stuff
\newcommand{\image}[3][1]{
    \begin{figure}[h]
        \centering
        \includegraphics[#1]{#2}
        \caption{#3}
        \label{#3}
    \end{figure}
}
%% [END] Figure related stuff

\begin{document}
\begin{titlepage}
    \centering
    \vfill
    {\bfseries\LARGE
        \vSubject\\
        \vskip0.25cm
        \vSubtitle
    }
    \vfill
    \includegraphics[width=6cm]{images/polinema-logo.png}
    \vfill
    {
        \textbf{Name}\\
        \vName\\
        \vskip0.5cm
        \textbf{NIM}\\
        \vNIM\\
        \vskip0.5cm
        \textbf{Class}\\
        \vClass\\
        \vskip0.5cm
        \textbf{Department}\\
        \vDepartment\\
        \vskip0.5cm
        \textbf{Study Program}\\
        \vStudyProgram
    }
\end{titlepage}

\section{Midterm Exam}

\large{\textbf{Case Study}}\\
You are a real estate businessman. You have several units of houses, land, and shophouses spread across Malang City, and property acquisition plans are seen
as future prospects. As a businessman your goal is to get good sales, and good profits.

To achieve that goal, you have several strategies that are divided into the following categories:
\begin{enumerate}
    \item Determine the best selling price for each unit
    \item Determine the target marketing on social media
    \item Choose the right location for the property unit to be acquired
    \item Determine the specifications for the house
    \item Choose the priority of the unit owned, between a house, land, or shophouse
\end{enumerate}
Your job is to choose one of the five stategies above.

\large{\textbf{Questions}}\\
\begin{enumerate}
    \item {
        From one of the strategies you choose, define a clear problem with machine learning as the solution approach

        We can choose the first strategy, which is to determine the best price of each unit. We can do this using machine learning
        by doing a regression analysis. We can use the data of the price of the unit and the specifications of the unit to determine
        the best price of the unit.
    }
    \item {
        Develop a data ingestion strategy, determine what data will be processed, where it will come from

        The data that will be processed is the price of the unit and the specifications of the unit. The data could come from multiple sources,
        such as the previous sales data, or the data from the competitors.
    }
    \item {
        Make a sample data, at least 20 items

        \footnotesize
        \begin{tabularx}{\columnwidth}{|X|X|X|X|X|X|X|}
            \hline
            \textbf{No} & \textbf{Price} & \textbf{House} & \textbf{Land} & \textbf{Bedroom} & \textbf{Bathroom} & \textbf{Garage}\\
            \hline
            1 & 3800000000 & 220 & 220 & 3 & 3 & 0\\
            \hline
            2 & 4600000000 & 180 & 137 & 4 & 3 & 2\\
            \hline
            3  & 3000000000 & 267 & 250 & 4 & 4 & 4\\
            \hline
            4 & 430000000 & 40 & 25 & 2 & 2 & 0 \\
            \hline
            5 & 9000000000 & 400 & 355 & 6 & 5 & 3 \\
            \hline
            6 & 4970000000 & 300 & 154 & 5 & 3 & 3 \\
            \hline
            7 & 2600000000 & 120 & 150 & 3 & 2 & 1 \\
            \hline
            8 & 10500000000 & 350 & 247 & 4 & 4 & 0 \\
            \hline
            9 & 3250000000 & 125 & 90 & 3 & 3 & 0 \\
            \hline
            10 & 4500000000 & 250 & 96 & 5 & 4 & 1 \\
            \hline
            11 & 3600000000 & 154 & 110 & 3 & 3 & 3 \\
            \hline
            12 & 9500000000 & 450 & 240 & 4 & 4 & 1 \\
            \hline
            13 & 10500000000 & 218 & 118 & 3 & 3 & 2 \\
            \hline
            14 & 12500000000 & 200 & 979 & 4 & 2 & 6 \\
            \hline
            15 & 4600000000 & 180 & 137 & 5 & 4 & 2 \\
            \hline
            16 & 3000000000 & 126 & 144 & 4 & 2 & 2 \\
            \hline
            17 & 6000000000 & 400 & 150 & 5 & 4 & 1 \\
            \hline
            18 & 7500000000 & 150 & 253 & 5 & 2 & 2 \\
            \hline
            19 & 18000000000 & 200 & 251 & 5 & 3 & 3 \\
            \hline
            20 & 9700000000 & 450 & 248 & 5 & 5 & 4 \\
            \hline
        \end{tabularx}
    }
    \item {
        What kind of machine learning model / method will be used?

        The model that we can use to work with this type of data or case study is the
        \textbf{Linear Regression} model. We can use this model to predict the price of the unit
        based on the specifications of the unit.
    }
    \item {
        Explain whether the method is supervised or unsupervised!

        The method being used is supervised learning, because we have the data of the previous sales
        that we can use to train the model. We can use the data to train the model to predict the best price
        of the house.
    }
    \item {
        Explain why do you choose this model!

        The model was chosen because it is the most suitable model to work with and it's also one of the simplest
        method. The model is also easy to understand and implement.
    }
    \pagebreak
    \item {
        Explain the machine learning method that will be used!

        The machine learning method that will be used is the \textbf{Linear Regression} method.
        Linear regression is a linear approach to modeling the relationship between a scalar response
        or dependent variables and one or more explanatory variables or independent variables.
        The case study that we have is a regression problem, because we want to predict the price of the unit
        based on the specifications of the unit.
    }
\end{enumerate}

\end{document}

