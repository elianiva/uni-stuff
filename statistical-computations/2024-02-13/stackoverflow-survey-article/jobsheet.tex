\documentclass[12pt,titlepage]{article}
\usepackage[margin=1.25in]{geometry}
\usepackage{graphicx,amsmath,color,soul}

%% Variables definition
\newcommand{\vSubject}{Statistical Computations}
\newcommand{\vSubtitle}{Stackoverflow Survey Article}
\newcommand{\vName}{Dicha Zelianivan Arkana}
\newcommand{\vNIM}{2241720002}
\newcommand{\vClass}{2i}
\newcommand{\vDepartment}{Information Technology}
\newcommand{\vStudyProgram}{D4 Informatics Engineering}

%% [START] Tikz related stuff
\usepackage{tikz}
\usetikzlibrary{svg.path,calc,shapes.geometric,shapes.misc}
\tikzstyle{terminator} = [rectangle, draw, text centered, rounded corners = 1em, minimum height=2em]
\tikzstyle{preparation} = [chamfered rectangle, chamfered rectangle sep=0.75em, draw, text centered, minimum height = 2em]
\tikzstyle{process} = [rectangle, draw, text centered, minimum height=2em]
\tikzstyle{decision} = [diamond, aspect=2, draw, text centered, minimum height=2em]
\tikzstyle{data}=[trapezium, draw, text centered, trapezium left angle=60, trapezium right angle=120, minimum height=2em]
\tikzstyle{connector} = [line width=0.25mm,->]
%% [END] Tikz related stuff

%% [START] Fancy header related stuff
\usepackage{fancyhdr}
\pagestyle{fancy}
\setlength{\headheight}{15pt} % compensate fancyhdr style
\fancyhead{}
\fancyfoot{}
\fancyfoot[L]{\thepage}
\fancyfoot[R]{\textit{\vSubject - \vSubtitle}}
\renewcommand{\footrulewidth}{0.4pt}% default is 0pt, overline for footer
%% [END] Fancy header related stuff

%% [START] Custom tabular command related stuff
\usepackage{tabularx}
\newcommand{\details}[2]{
    #1 & #2  \\
}
%% [END] Custom tabular command related stuff

%% [START] Figure related stuff
\newcommand{\image}[3][1]{
    \begin{figure}[h]
        \centering
        \includegraphics[#1]{#2}
        \caption{#3}
        \label{#3}
    \end{figure}
}
%% [END] Figure related stuff

\begin{document}
\begin{titlepage}
    \centering
    \vfill
    {\bfseries\LARGE
        \vSubject\\
        \vskip0.25cm
        \vSubtitle
    }
    \vfill
    \includegraphics[width=6cm]{images/polinema-logo.png}
    \vfill
    {
        \textbf{Name}\\
        \vName\\
        \vskip0.5cm
        \textbf{NIM}\\
        \vNIM\\
        \vskip0.5cm
        \textbf{Class}\\
        \vClass\\
        \vskip0.5cm
        \textbf{Department}\\
        \vDepartment\\
        \vskip0.5cm
        \textbf{Study Program}\\
        \vStudyProgram
    }
\end{titlepage}

\section{Overview}
Stackoverflow is a website that is used by programmers to ask questions and get answers from other people.
Each year, Stackoverflow conducts a survey to get insights about the developer community. The survey
covers a wide range of topics, including programming languages, tools, and technologies, as well as
demographics and employment information. This article provide some analysis of the results from the
2023 Stackoverflow survey and discuss some of the key findings. This article will not cover
the full scope of the survey, but hopefully it should be enough to give the readers an idea
of how the software engineering scene in general.

\section{Methodology}
The data that are used in this article are taken from the Stackoverflow survey results. The survey
result can be found at \texttt{https://survey.stackoverflow.co/2023/}. The survey was conducted from
8 May 2023 to 19 May 2023 with 89,184 software developers from 185 countries. Although, approximately 
2.000 responses were not included in this analysis because they were deemed incomplete or other
reasons.

The median time spent on the survey for qualified responses was almost 18 minutes. Many questions
were only shown to respondents based on their previous answers. The questions were organized into
several blocks of questions, which were randomized in order.

\section{Developer Profile and Demographics}

\subsection{Education}
Based on the survey results, the majority of developers have a bachelor's degree followed by having
a master's degree. Only a small percentage of the respondents have a doctoral degree, even less
than those who are still in secondary school.

\subsection{Experience}
Almost half of the respondents have been coding less than ten years. Most of the respondents have
5-9 years of coding experience, around 26\% of the respondents. On average, Australia has the most
years of coding experience, with 17.54 years, while India has the least, with 7.79 years. Most
higher ups position such as C-suite, VP, etc have the most years of coding experience averaging
around 17.43 years.

\subsection{Age}
\sethlcolor{cyan}
43\% of Professional Developers are 25-34 years old. But more than half of the respondents
learning to code are 18-24 years old.

\subsection{Gender}
Interestingly, Stackoverflow survey in 2023 didn't include the question for gender.
This is probably due to fact that in recent years there has been a lot of controversy
regarding gender identity and such.

\subsection{Conclusion}
Based on the survey, the majority of developers have a bachelor's degree and have been coding
for less than ten years. \hl{The majority of professional developers are 25-34 years old}, and the
survey didn't include the question for gender identity. \sethlcolor{yellow} \hl{We can conclude that the majority of
developers are young and have a relatively short amount of experience.}

\section{Technologies}
\subsection{Languages}
2023 continues JavaScript’s streak as its eleventh year in a row as the most commonly-used
programming language having 63.61\% responses, followed by HTML/CSS as markup language.
Python is the third most used language, overtake SQL which is the fourth most used language.
The big mover, gaining seven spots since 2022 was Lua, an embeddable scripting language.

Rust while only being the 14th most used language, is the most admired language, with 84.66\% of
the respondents showed their admiration for the language. Compare this to the least admired 
language: MATLAB. Less than 20\% of developers who used this language want to use it again next year.

Zig is the highest-paid language to know this year (a new addition), while Clojure gets knocked
from the top spot with a 10\% decrease from 2022. The correlation of the language with the highest
average salary with it being popular is not very strong, as Zig is only the 41th most popular language.
This is probably due to the fact that Zig is a relatively new language and not many developers know it
or able to use it professionally.

Do bare in mind that the percentages are not mutually exclusive, meaning that a respondent
can choose multiple languages. This is why the percentages can add up to more than 100\%.

\subsection{Databases}
PostgresQL with 45.55\% has taken the top spot as the most popular database, overtaking 
MySQL in previous years. MongoDB being the top NoSQL database, followed by Redis and Firebase.

There doesn't seem to be a significant variant in the database usage based on the result
of the survey. The top 10 databases are still the same as the previous years. The pattern
observed is that most people prefer to use a stable and proven to be reliable databases over
the ones that are still new and trying to make its way into the market.

\sethlcolor{cyan}
\hl{PostgreSQL, Redis, and Datomic are the most admired databases with 71.23\%, 69.92\%, and 70.49\%
consequently.} \sethlcolor{yellow} \hl{Interestingly, Datomic have the least users among the three of them.
That kind of admiration should push others to consider Datomic as a viable option.}

\subsection{Cloud Platforms}
AWS remains the most used cloud platform for all respondents. AWS handily makes it to the 
top spot, almost doubling the percentange of the second most used cloud platform for all
respondents, Azure, with them having 48.62\% and 26.03\% respectively.

\sethlcolor{cyan}
\hl{Despite that, cloud platform with the largest proportion that the respondents have used and want
to continue using them are Hetzner and Vercel (69\%+);} more developers would choose to work with
these two cloud platforms over those that would choose to and have worked with the top three
(AWS, Azure, and Google Cloud). \sethlcolor{yellow} \hl{This is a very interesting result, as it shows that the top
three cloud platforms are not necessarily the most preferred ones.}

\section{Conclusion}
Based on the survey, we can see that the trend of the respondents are mostly young and have
a relatively short amount of experience. \hl{We can conclude that most of them are working
for the web development industry, as the most used programming language is JavaScript and
HTML/CSS being the second most used language.}

This article only covers a small portion of the survey results and in no way represents the
full scope of the survey. The purpose of the article is to provide the readers an idea
on how the software engineering scene is in 2023. The full survey results can be
found at \texttt{https://insights.stackoverflow.com/survey/2023}.

\section{Author's Note}

The reason why I bring up the topic is because the result of Stackoverflow survey always
provides an interesting insight into the software engineering industry. It's always interesting
to see how the industry changes from year to year and how the trends evolve. The survey
results can be used to make informed decisions about the technologies and tools to learn
and use in the future based on other people experiences.

Also, note that some of the text in this article are highlighted with colours, cyan to indicate that
it's a descriptive statistics and yellow to indicate that it's an inferential statistics. Although,
only a few of them are marked so that the reading experience is not disrupted.

(((This is only because I made this article for my statistics class assignment. Hence, I need to put
the colours to indicate which part of the text is descriptive and which part is inferential.
An actual article wouldn't have this kind of colouring.)))

\end{document}

