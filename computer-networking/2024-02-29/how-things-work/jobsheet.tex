\documentclass[12pt,titlepage]{article}
\usepackage[margin=1.25in]{geometry}
\usepackage{graphicx,amsmath,blindtext}

%% Variables definition
\newcommand{\vSubject}{Computer Networking}
\newcommand{\vSubtitle}{How things work}
\newcommand{\vName}{Dicha Zelianivan Arkana}
\newcommand{\vNIM}{2241720002}
\newcommand{\vClass}{2i}
\newcommand{\vDepartment}{Information Technology}
\newcommand{\vStudyProgram}{D4 Informatics Engineering}

%% [START] Tikz related stuff
\usepackage{tikz}
\usetikzlibrary{svg.path,calc,shapes.geometric,shapes.misc}
\tikzstyle{terminator} = [rectangle, draw, text centered, rounded corners = 1em, minimum height=2em]
\tikzstyle{preparation} = [chamfered rectangle, chamfered rectangle sep=0.75em, draw, text centered, minimum height = 2em]
\tikzstyle{process} = [rectangle, draw, text centered, minimum height=2em]
\tikzstyle{decision} = [diamond, aspect=2, draw, text centered, minimum height=2em]
\tikzstyle{data}=[trapezium, draw, text centered, trapezium left angle=60, trapezium right angle=120, minimum height=2em]
\tikzstyle{connector} = [line width=0.25mm,->]
%% [END] Tikz related stuff

%% [START] Fancy header related stuff
\usepackage{fancyhdr}
\pagestyle{fancy}
\setlength{\headheight}{15pt} % compensate fancyhdr style
\fancyhead{}
\fancyfoot{}
\fancyfoot[L]{\thepage}
\fancyfoot[R]{\textit{\vSubject - \vSubtitle}}
\renewcommand{\footrulewidth}{0.4pt}% default is 0pt, overline for footer
%% [END] Fancy header related stuff

%% [START] Custom tabular command related stuff
\usepackage{tabularx}
\newcommand{\details}[2]{
    #1 & #2  \\
}
%% [END] Custom tabular command related stuff

%% [START] Figure related stuff
\newcommand{\image}[3][1]{
    \begin{figure}[h]
        \centering
        \includegraphics[#1]{#2}
        \caption{#3}
        \label{#3}
    \end{figure}
}
%% [END] Figure related stuff

\begin{document}
\begin{titlepage}
    \centering
    \vfill
    {\bfseries\LARGE
        \vSubject\\
        \vskip0.25cm
        \vSubtitle
    }
    \vfill
    \includegraphics[width=6cm]{images/polinema-logo.png}
    \vfill
    {
        \textbf{Name}\\
        \vName\\
        \vskip0.5cm
        \textbf{NIM}\\
        \vNIM\\
        \vskip0.5cm
        \textbf{Class}\\
        \vClass\\
        \vskip0.5cm
        \textbf{Department}\\
        \vDepartment\\
        \vskip0.5cm
        \textbf{Study Program}\\
        \vStudyProgram
    }
\end{titlepage}

\tableofcontents

\pagebreak

\section{Introduction}

The internet that we use everyday comprises of many different devices and technologies. Each of those
devices communicate through different types of protocols, which have they're own purpose and use case.
This article will explain how those devices communicate with each other through the aforementioned protocols.

This article is in no way tries to be a comprehensive guide over these protocols. It should serve as an introduction
to the topic and should be used as a starting point for further research. Some of the protocols
that will be discussed in this article are:
\begin{itemize}
    \item DNS
    \item HTTP
    \item Email Services (SMTP / POP3 / IMAP)
    \item FTP
    \item DHCP
    \item Telnet/SSH Services
\end{itemize}

\section{Domain Name System (DNS)}
Domain Name System or commonly abbreviated as DNS is a system that translates domain names to IP addresses.
Think of it like a phonebook where it converts a name to a phone number. This system is crucial for the internet
to function as it is today. Without DNS, we would have to remember the IP address of every website we want to visit.

In order to understand the process behind the DNS resolution, it’s important to learn about
the different hardware components a DNS query must pass between. For the web browser, the DNS
lookup occurs "behind the scenes" and requires no interaction from the user’s computer apart from
the initial request.

The process of resolving a domain name involves the following steps:
\begin{enumerate}
    \item {
        A user types ‘example.com’ into a web browser and the query travels into the Internet and is received by a DNS recursive resolver.
    }
    \item {
        The resolver then queries a DNS root nameserver (.).
    }
    \item {
        The root server then responds to the resolver with the address of a Top Level Domain (TLD) DNS
        server (such as .com or .net), which stores the information for its domains. When searching for
        example.com, our request is pointed toward the .com TLD.
    }
    \item {
        The resolver then makes a request to the .com TLD.
    }
    \item {
        The TLD server then responds with the IP address of the domain’s nameserver, example.com.
    }
    \item {
        Lastly, the recursive resolver sends a query to the domain’s nameserver.
    }
    \item {
        The IP address for example.com is then returned to the resolver from the nameserver.
    }
    \item {
        The DNS resolver then responds to the web browser with the IP address of the domain requested initially.
    }
\end{enumerate}

Once the 8 steps of the DNS lookup have returned the IP address for example.com, the browser is able to make the request for the web page:

\begin{enumerate}
    \item {
        The browser makes a HTTP request to the IP address.
        }
    \item {
        The server at that IP returns the webpage to be rendered in the browser.
    }
\end{enumerate}

All of these steps happen in a fraction of a second. The DNS system is a critical component of the Internet infrastructure, providing a way
for your computer to look up the addresses of the servers where your requested pages are hosted.

\section{Hypertext Transfer Protocol (HTTP)}

Hypertext Transfer Protocol, more commonly known as HTTP is probably the most widely known protocol on the Internet. It's
basically the foundation of World Wide Web, and is used to load webpages using hypertext links. It operates in the application layer protocol
designed to transfer information between networked devices on top of other layers of network protocol stack.
A typical flow of HTTP communication involves a client and server devices.

HTTP follows a classical client-server model, with a client opening a connection to make a request,
then waiting until it receives a response. HTTP is a stateless protocol, meaning that the server does not
keep any data (state) between two requests.

When we make a HTTP request to a server, it typically contains several things:
\begin{enumerate}
    \item {
        \textbf{HTTP Version Type} - As of the time when this article was written, there are 3 versions of HTTP that are widely used.
        Namely, HTTP/1.1, HTTP/2.0, and HTTP/3.0
    }
    \item {
        \textbf{URL} - Uniform Resource Locator (URL) is an address to a resource on the Internet.
        It can be  webpage, image, video, or any other file.
    }
    \item {
        \textbf{HTTP Method} - Also known as HTTP verb. Each request that performs different action
        carries different HTTP verb. The most common HTTP verbs are GET, POST, PUT, DELETE, and PATCH.
    }
    \item {
        \textbf{HTTP Request Headers} - These are key-value pairs that are sent to the server along with the request.
        This part functions as a way to send additional information to the server that acts like a metadata such as the user agent,
        format of the data being sent, etc.
    }
    \item {
        \textbf{HTTP Request Body (optional)} - This part is optional and is only used when the request method is POST, PUT, or PATH.
        The body of an HTTP request contains any information being submitted to the web server, such as a username and password, or any other data entered into a form.
    }
\end{enumerate}

After making a request, we usually get a HTTP response back from the server that consists of:
\begin{enumerate}
    \item {
        \textbf{HTTP Status Code} - It's a 3 digit numbers used to indicate whether a HTTP request
        is successful or not. They are broken into 5 blocks:
        \begin{itemize}
            \item \textbf{1xx} - Informational
            \item \textbf{2xx} - Success
            \item \textbf{3xx} - Redirection
            \item \textbf{4xx} - Client Error
            \item \textbf{5xx} - Server Error
        \end{itemize}
        The xx refers to different numbers between 00 - 99.
    }
    \item {
        \textbf{HTTP Response Headers} - Much like its request counterparts, these are key-value pairs that are
        sent from the server to the client and it serves pretty much the same purpose as the request headers.
    }
    \item {
        \textbf{HTTP Response Body} - It contains the actual content that the server sends back to the client.
        Most commonly it's a HTML file, but it can also be a JSON, XML, or any other format.
    }
\end{enumerate}

\section{Email Services (SMTP / POP3 / IMAP)}

\subsection{Simple Mail Transfer Protocol (SMTP)}

SMTP stands for Simple Mail Transfer Protocol. It is a communication protocol used for sending
and receiving email messages over the Internet. Mail servers and other message transfer agents (MTAs)
use SMTP to send, receive and relay mail messages.

As the “T” in its name indicates, SMTP is a transport protocol: All it does is move messages from point A to point B.

Like many other Internet protocols, SMTP is intended to be used on top of the Transmission Control Protocol (TCP), which guarantees reliable delivery of the individual packets that make up a conversation.

The use of IP means that we can depend on SMTP to eventually get the message contents to a server, but what happens to it after that is up to the server.

Imagine what happens when a delivery driver leaves a package at your office on a Saturday. What happens to it after delivery isn’t his problem.

Deliverability, monitoring, tracking, authentication, and encryption are all examples of services that SMTP doesn’t necessarily provide itself but that are still very valuable.

\subsection{P}


\end{document}

