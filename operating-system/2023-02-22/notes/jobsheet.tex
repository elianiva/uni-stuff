\documentclass[12pt,titlepage]{article}
\usepackage[margin=1.25in]{geometry}
\usepackage{graphicx,amsmath,blindtext}

%% Variables definition
\newcommand{\vSubject}{Operating System}
\newcommand{\vSubtitle}{Notes}
\newcommand{\vName}{Dicha Zelianivan Arkana}
\newcommand{\vNIM}{2241720002}
\newcommand{\vClass}{1i}
\newcommand{\vDepartment}{Information Technology}
\newcommand{\vStudyProgram}{D4 Informatics Engineering}

%% [START] Tikz related stuff
\usepackage{tikz}
\usetikzlibrary{svg.path,calc,shapes.geometric,shapes.misc}
\tikzstyle{terminator} = [rectangle, draw, text centered, rounded corners = 1em, minimum height=2em]
\tikzstyle{preparation} = [chamfered rectangle, chamfered rectangle sep=0.75em, draw, text centered, minimum height = 2em]
\tikzstyle{process} = [rectangle, draw, text centered, minimum height=2em]
\tikzstyle{decision} = [diamond, aspect=2, draw, text centered, minimum height=2em]
\tikzstyle{data}=[trapezium, draw, text centered, trapezium left angle=60, trapezium right angle=120, minimum height=2em]
\tikzstyle{connector} = [line width=0.25mm,->]
%% [END] Tikz related stuff

%% [START] Fancy header related stuff
\usepackage{fancyhdr}
\pagestyle{fancy}
\setlength{\headheight}{15pt} % compensate fancyhdr style
\fancyhead{}
\fancyfoot{}
\fancyfoot[L]{\thepage}
\fancyfoot[R]{\textit{\vSubject - \vSubtitle}}
\renewcommand{\footrulewidth}{0.4pt}% default is 0pt, overline for footer
%% [END] Fancy header related stuff

%% [START] Custom tabular command related stuff
\usepackage{tabularx}
\newcommand{\details}[2]{
    #1 & #2  \\
}
%% [END] Custom tabular command related stuff

%% [START] Figure related stuff
\newcommand{\image}[3][1]{
    \begin{figure}[h]
        \centering
        \includegraphics[#1]{#2}
        \caption{#3}
        \label{#3}
    \end{figure}
}
%% [END] Figure related stuff

\begin{document}
\begin{titlepage}
    \centering
    \vfill
    {\bfseries\LARGE
        \vSubject\\
        \vskip0.25cm
        \vSubtitle
    }
    \vfill
    \includegraphics[width=6cm]{images/polinema-logo.png}
    \vfill
    {
        \textbf{Name}\\
        \vName\\
        \vskip0.5cm
        \textbf{NIM}\\
        \vNIM\\
        \vskip0.5cm
        \textbf{Class}\\
        \vClass\\
        \vskip0.5cm
        \textbf{Department}\\
        \vDepartment\\
        \vskip0.5cm
        \textbf{Study Program}\\
        \vStudyProgram
    }
\end{titlepage}

\tableofcontents

\newpage

\section{Introduction to Operating System}
\subsection{What is an Operating System}
\begin{itemize}
    \item Acts as an intermediary between a user of a computer and the computer hardware
    \item {
        \textbf{Goals}
        \begin{itemize}
            \item Execute user programs and make solving user problem easier
            \item Make the computer system convenient to use
            \item Use the computer hardware in an efficient manner
        \end{itemize}
    }
\end{itemize}

\subsection{Computer System Structure}
\begin{itemize}
    \item \textbf{Hardware}: provides basic computing resources
    \item \textbf{Operating System}: Controls and coordinates use of hardware among various application and users
    \item \textbf{Application Programs}: Define the ways in which the system resources are used to solve the computing problems of the users
    \item \textbf{User}: The one who operates the computer
\end{itemize}

\subsection{Operating System Definition}
\begin{itemize}
    \item \textbf{Resource Allocator}: Manages and controls all hardware resources
    \item \textbf{Control Program}: Controles the execution of programs to prevent errors and improper use of the computer 
\end{itemize}

\subsection{Types of Operating System}
\begin{itemize}
    \item Batching System
    \item Multi-programming System
    \item Time-sharing / Multi-tasking System
    \item Multiprocessing System (Asymmetry \& Symmetry)
\end{itemize}

\end{document}

