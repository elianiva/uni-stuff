\documentclass[12pt,titlepage]{article}
\usepackage[margin=1.25in]{geometry}
\usepackage{graphicx,amsmath,blindtext,hyperref}

%% Variables definition
\newcommand{\vSubject}{Islam}
\newcommand{\vSubtitle}{Human Being and Religion}
\newcommand{\vName}{Dicha Zelianivan Arkana}
\newcommand{\vNIM}{2241720002}
\newcommand{\vClass}{1i}
\newcommand{\vDepartment}{Information Technology}
\newcommand{\vStudyProgram}{D4 Informatics Engineering}

%% [START] Tikz related stuff
\usepackage{tikz}
\usetikzlibrary{svg.path,calc,shapes.geometric,shapes.misc}
\tikzstyle{terminator} = [rectangle, draw, text centered, rounded corners = 1em, minimum height=2em]
\tikzstyle{preparation} = [chamfered rectangle, chamfered rectangle sep=0.75em, draw, text centered, minimum height = 2em]
\tikzstyle{process} = [rectangle, draw, text centered, minimum height=2em]
\tikzstyle{decision} = [diamond, aspect=2, draw, text centered, minimum height=2em]
\tikzstyle{data}=[trapezium, draw, text centered, trapezium left angle=60, trapezium right angle=120, minimum height=2em]
\tikzstyle{connector} = [line width=0.25mm,->]
%% [END] Tikz related stuff

%% [START] Fancy header related stuff
\usepackage{fancyhdr}
\pagestyle{fancy}
\setlength{\headheight}{15pt} % compensate fancyhdr style
\fancyhead{}
\fancyfoot{}
\fancyfoot[L]{\thepage}
\fancyfoot[R]{\textit{\vSubject - \vSubtitle}}
\renewcommand{\footrulewidth}{0.4pt}% default is 0pt, overline for footer
%% [END] Fancy header related stuff

%% [START] Custom tabular command related stuff
\usepackage{tabularx}
\newcommand{\details}[2]{
    #1 & #2  \\
}
%% [END] Custom tabular command related stuff

%% [START] Figure related stuff
\newcommand{\image}[3][1]{
    \begin{figure}[h]
        \centering
        \includegraphics[#1]{#2}
        \caption{#3}
        \label{#3}
    \end{figure}
}
%% [END] Figure related stuff

\begin{document}

\begin{titlepage}
    \centering
    \vfill
    {\bfseries\LARGE
        \vSubject\\
        \vskip0.25cm
        \vSubtitle
    }
    \vfill
    \includegraphics[width=6cm]{images/polinema-logo.png}
    \vfill
    {
        \textbf{Name}\\
        \vName\\
        \vskip0.5cm
        \textbf{NIM}\\
        \vNIM\\
        \vskip0.5cm
        \textbf{Class}\\
        \vClass\\
        \vskip0.5cm
        \textbf{Department}\\
        \vDepartment\\
        \vskip0.5cm
        \textbf{Study Program}\\
        \vStudyProgram
    }
\end{titlepage}

\renewcommand{\thesection}{\Alph{section}.}

\section{Introduction of Surah Al-Mulk}
\begin{itemize}
    \item It's better if we look to the surah Al-Mulk when we're studying about Human Being and Religion
    \item {
        There are mentions of surah Al-Mulk in several hadith, namely:
        \begin{itemize}
            \item {
                “A surah from the Quran containing thirty verses will intercede for a man 
                so that he will be forgiven. It is the surah Tabarak Alladhi bi yadihi’l-mulk 
                [i.e.,Surat al-Mulk].” (Narrated by al-Tirmidhi, 2891; Abu Dawud, 1400; Ibn Majah, 3786.
            }
            \item {
                Abd-Allah ibn Mas'ud said: Whoever reads Tabarak alladhi bi
                yadihi'l-mulk [i.e., Surat al-Mulk] every night, Allah will protect him from the torment
                of the grave . At the time of the Messenger of Allah (peace and blessings of Allah be
                upon him) we used to call it Al-Mani'ah (that which protects).
            }
        \end{itemize}
    }
\end{itemize}

\section{Contents of Surah Al-Mulk}
Surah Al Mulk can be categorised into several parts where each part focuses on its own topic.
\begin{enumerate}
    \item Allah's power (1 - 4)
    \item Heaven, Hell (5 - 15)
    \item Sudden Threat (16 - 22)
    \item Life is Short (23 - 24)
    \item When is the time for a judgement day (25 - 27)
    \item Human is weak (28 - 30)
\end{enumerate}

\section{The purposes of understanding Surah Al-Mulk}
\begin{enumerate}
    \item Understanding the purpose of human life in this world
    \item As a guidance for living
    \item Knowing the concept of khusnul khotimah
\end{enumerate}

\pagebreak

\section{Al-Mulk Verse 3}
\begin{figure}[h]
    \includegraphics[width=\textwidth]{images/verse-3.png}
\end{figure}
\textit{
    ``He is the One'' Who created seven heavens, one above the other.
    You will never see any imperfection in the creation of the Most Compassionate.
    So look again: do you see any flaws?
}

\vspace{5mm}

It states that Allah SWT created life and death to test us. In this lifetime Allah doesn’t
demand the perfection but He demanded for us to be better.
If we continue to be better than before, to be better than yesterday, InsyaAllah we will achieve the khusnul khotimah.

\section{The discoveries of 7 Heaven or Skies according to The Quran}
\begin{itemize}
    \item There are many new discoveries that has been mentioned in The Quran 1400 years ago
    \item {
        One of them is the concept of 7 layers of skies, namely:
        \begin{enumerate}
            \item Troposphere
            \item Stratosphere
            \item Mesosphere
            \item Thermosphere
            \item Exosphere
            \item Ozonosphere
            \item Ionosphere
        \end{enumerate}
    }
    \item This proves that The Quran contains facts and there is no doubt about it
    \item It also serves a reason for us to believe in religion because it provides guidance to live our life
\end{itemize}

\pagebreak

\section{Conclusion}
\begin{itemize}
    \item Surah Al-Mulk is one of the special surah in The Quran that can save us from the torment of grave
    \item The authority and the kingdom of Allah must be blessed
    \item The true life of a believer is in the hereafter
    \item We live in this world in order to be tested
    \item The Quran is full of truth and facts and there is no doubt about it
\end{itemize}

\hspace{2cm}

Source material: \underline{\href{https://drive.google.com/file/d/1_QuDzzcJptFiAH-yyABbWV-5_B8Z4_oW/view}{Video on Google Drive}}.

\end{document}

